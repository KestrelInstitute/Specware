% Reference Card for the ilisp Interface, as reported in ilisp-5.8.
% Last edited: Sat Sep 27 20:07:53 1997 by vkyr (Valentino Kyriakides)

%**start of header
\newcount\columnsperpage

% This file can be printed with 1, 2, or 3 columns per page (see below).
% Specify how many you want here.  Nothing else needs to be changed.

\columnsperpage=3

% Copyright (c) 1997 by no one.  e.g. distribute it at will, subject to
% permissions below.
%
% Author: (e.g. please send obvious bugs to)
%  Valentino Kyriakides
%  Internet: kyriakides@lavielle.com
%
% Of course, no one makes any claims on this, nor do I promise to "support" it.
% Thanks to Stephen Gildea, Paul Rubin, Bob Chassell, Len Tower,
% and Richard Mlynarik for the excellent GNU Emacs reference card on which
% this is based.
%
% The idea of this was to have a piece of paper that prods your memory if
% you've forgotten the name of a function, a keybinding, etc.  It assumes
% pretty serious familiarity with the ilisp manual, but hopefully saves the effort
% (in conjunction with all the manuals) of lugging out a
% printed copy in most cases.  It of course reflects my (the author's)
% biases about what was likely to be interesting or used often.
%
% To view, do 'tex ilisp-refcard.tex'. 
% To print if you use columnsperpage=3, you must use have a dvips-like
% program that prints in landscape format!

% This file is distributed in the hope that it will be useful, but WITHOUT
% ANY WARRANTY.  No author or distributor accepts responsibility to anyone
% for the consequences of using it or for whether it serves any particular
% purpose or describes any piece of software unless they say so in writing.  
%
% Permission is granted to copy, modify and redistribute this source
% file provided the permission notices are preserved on all copies.
%
% Permission is granted to process this file with TeX and print the results.

% This file is intended to be processed by plain TeX (TeX82).
%
% The final reference card has six columns, three on each side.
% This file can be used to produce it in any of three ways:
% 1 column per page
%    produces six separate pages, each of which needs to be reduced to 80%.
%    This gives the best resolution.
% 2 columns per page
%    produces three already-reduced pages.
%    You will still need to cut and paste.
% 3 columns per page
%    produces two pages which must be printed sideways to make a
%    ready-to-use 8.5 x 11 inch reference card.
%    For this you need a dvi device driver that can print sideways.
% Which mode to use is controlled by setting \columnsperpage above.

\def\versionnumber{1.0}
\def\year{1997}
\def\version{September \year\ v\versionnumber}

\def\shortcopyrightnotice{\vskip 1ex plus 2 fill \centerline{\small September
  \year by Valentino Kyriakides.  Permissions on back.  v\versionnumber}}

\def\copyrightnotice{
\vskip 1ex plus 2 fill\begingroup\small
\centerline{Designed by Valentino Kyriakides (kyriakides@lavielle.com), 9/97.}
\centerline{Permission is granted to make \& distribute copies}
\centerline{of this card provided this permission notice is preserved.}
\endgroup}

% ===================================================================
% Macros to define this "mode".
% ===================================================================
%
% make \bye not \outer so that the \def\bye in the \else clause below
% can be scanned without complaint.
\def\bye{\par\vfill\supereject\end}

\newdimen\intercolumnskip
\newbox\columna
\newbox\columnb

\def\ncolumns{\the\columnsperpage}

\message{[\ncolumns\space 
  column\if 1\ncolumns\else s\fi\space per page]}

\def\scaledmag#1{ scaled \magstep #1}

% This multi-way format was designed by Stephen Gildea October 1986.
\if 1\ncolumns
  \hsize 4in
  \vsize 10in
  \voffset -.7in
  \font\titlefont=\fontname\tenbf \scaledmag4
  \font\secfont=\fontname\tenbf \scaledmag3
  \font\subsecfont=\fontname\tenbf \scaledmag2
  \font\smallfont=\fontname\sevenrm
  \font\smallsy=\fontname\sevensy

  \footline{\hss\folio}
  \def\makefootline{\baselineskip10pt\hsize6.5in\line{\the\footline}}
\else
  \hsize 3.2in
  \vsize 7.95in
  \hoffset -.75in
  \voffset -.745in
  \font\titlefont=cmbx10 \scaledmag3
  \font\secfont=cmbx10 \scaledmag2
  \font\subsecfont=cmbx10 \scaledmag1
  \font\smallfont=cmr6
  \font\smallsy=cmsy6
  \font\eightrm=cmr8
  \font\eightbf=cmbx8
  \font\eightit=cmti8
  \font\eighttt=cmtt8
  \font\eightsy=cmsy8
  \textfont0=\eightrm
  \textfont2=\eightsy
  \def\rm{\eightrm}
  \def\bf{\eightbf}
  \def\it{\eightit}
  \def\tt{\eighttt}
  \normalbaselineskip=.8\normalbaselineskip
  \normallineskip=.8\normallineskip
  \normallineskiplimit=.8\normallineskiplimit
  \normalbaselines\rm		%make definitions take effect

  \if 2\ncolumns
    \let\maxcolumn=b
    \footline{\hss\rm\folio\hss}
    \def\makefootline{\vskip 2in \hsize=6.86in\line{\the\footline}}
  \else \if 3\ncolumns
    \let\maxcolumn=c
    \nopagenumbers
  \else
    \errhelp{You must set \columnsperpage equal to 1, 2, or 3.}
    \errmessage{Illegal number of columns per page}
  \fi\fi

  \intercolumnskip=.46in
  \def\abc{a}
  \output={%
      % This next line is useful when designing the layout.
      %\immediate\write16{Column \folio\abc\space starts with \firstmark}
      \if \maxcolumn\abc \multicolumnformat \global\def\abc{a}
      \else\if a\abc
	\global\setbox\columna\columnbox \global\def\abc{b}
        %% in case we never use \columnb (two-column mode)
        \global\setbox\columnb\hbox to -\intercolumnskip{}
      \else
	\global\setbox\columnb\columnbox \global\def\abc{c}\fi\fi}
  \def\multicolumnformat{\shipout\vbox{\makeheadline
      \hbox{\box\columna\hskip\intercolumnskip
        \box\columnb\hskip\intercolumnskip\columnbox}
      \makefootline}\advancepageno}
  \def\columnbox{\leftline{\pagebody}}

  \def\bye{\par\vfill\supereject
    \if a\abc \else\null\vfill\eject\fi
    \if a\abc \else\null\vfill\eject\fi
    \end}  
\fi

% we won't be using math mode much, so redefine some of the characters
% we might want to talk about
\catcode`\^=12
\catcode`\_=12

\chardef\\=`\\
\chardef\{=`\{
\chardef\}=`\}

\hyphenation{mini-buf-fer}

\parindent 0pt
\parskip 1ex plus .5ex minus .5ex

\def\small{\smallfont\textfont2=\smallsy\baselineskip=.8\baselineskip}

\outer\def\newcolumn{\vfill\eject}

%\outer\def\title#1{{\titlefont\centerline{#1}}\vskip 1ex plus .5ex}

%% KOT redefined a bit
\outer\def\title#1#2{{\titlefont\centerline{#1}}{\titlefont\centerline{#2}}\vskip 1ex plus .5ex}

%\outer\def\section#1{\par\filbreak
%  \vskip 3ex plus 2ex minus 2ex {\subsecfont #1}\mark{#1}%
%  \vskip 2ex plus 1ex minus 1.5ex}

%% KOT redefined to cram more on
\outer\def\section#1{\vfill\par\filbreak
  \vskip 3.0ex plus 0.95ex minus 1.7ex {\secfont #1}\mark{#1}%
  \vskip 1.8ex plus 0.9ex minus 1.15ex}

\outer\def\subsection#1{\vfill\par\filbreak
  \vskip 3.0ex plus 0.95ex minus 1.7ex {\subsecfont #1}\mark{#1}%
  \vskip 1.8ex plus 0.9ex minus 1.15ex}

\outer\def\shortsubsection#1{\vfill\par\filbreak
  \vskip 1.0ex plus 0.95ex minus 1.7ex {\subsecfont #1}\mark{#1}%
  \vskip 0.8ex plus 0.9ex minus 1.15ex}

\newdimen\keyindent

\def\beginindentedkeys{\keyindent=1em}
\def\endindentedkeys{\keyindent=0em}
\endindentedkeys

\def\paralign{\vskip\parskip\halign}

\def\<#1>{$\langle${\rm #1}$\rangle$}

\def\kbd#1{{\tt#1}\null}	%\null so not an abbrev even if period follows

\def\beginexample{\par\leavevmode\begingroup
  \obeylines\obeyspaces\parskip0pt\tt}
{\obeyspaces\global\let =\ }
\def\endexample{\endgroup}

\def\key#1#2{\leavevmode\hbox to \hsize{\vtop
  {\hsize=.79\hsize\rightskip=1em
  \hskip\keyindent\relax#1}\kbd{#2}\hfil}}

\def\com#1#2{\leavevmode\hbox to \hsize{\vtop
  {\hsize=.306\hsize\rightskip=1em
  \hskip\keyindent\relax#1}\kbd{#2}\hfil}}

\def\lcom#1#2{\leavevmode\hbox to \hsize{\vtop
  {\hsize=.49\hsize\rightskip=1em
  \hskip\keyindent\relax#1}\kbd{#2}\hfil}}


\newbox\metaxbox
\setbox\metaxbox\hbox{\kbd{M-x }}
\newdimen\metaxwidth
\metaxwidth=\wd\metaxbox

\def\metax#1#2{\leavevmode\hbox to \hsize{\hbox to .75\hsize
  {\hskip\keyindent\relax#1\hfil}%
  \hskip -\metaxwidth minus 1fil
  \kbd{#2}\hfil}}

\def\threecol#1#2#3{\hskip\keyindent\relax#1\hfil&\kbd{#2}\quad
  &\kbd{#3}\quad\cr}

%**end of header

% ===================================================================
% Real refcard related things start here.
% ===================================================================
\title{ilisp/Emacs}{Interface Reference Card}

\centerline{(for Gnu Emacs , XEmacs)}

This reference card is a quick-and-dirty guide to those parts of the
ilisp manual, which are likely to be used often.
It includes most of the ilisp Interface to emacs, plus many of the
extensions defined in the {\it ilisp online help}.  It is {\it not} a
substitute for the ilisp online manual.

\section{The ilisp/Emacs Interface }

You can use the ilisp interface from either GNU Emacs or XEmacs.  
Put these lines in your {\tt .emacs} file for clisp and allegro:

\kbd{(setq load-path (cons (expand-file-name }

\hskip 25pt \kbd{"/usr/local/lib/ilisp-5.8/") load-path))}
	 
\kbd{(require 'completer)}

\kbd{(autoload 'clisp "ilisp" } 

\hskip 25pt \kbd{"Inferior generic Common LISP." t)}

\kbd{(setq clisp-program "/usr/local/bin/clisp -I")}

\kbd{(autoload 'allegro "ilisp" }

\hskip 25pt \kbd{ "Inferior Allegro Common LISP." t)}

\kbd{(setq allegro-program "/usr/local/bin/cl")}

Then invoke lisp with either \kbd{M-x clisp}, 
or with \kbd{M-x allegro}.

`C-z' is the prefix-key for most ILISP commands. This can be changed by 
setting the variable `ilisp-prefix'.

For online help with emacs, use \kbd{C-h f}, \kbd{describe-function}.

\shortsubsection{Interrupts, aborts and errors}

If you want to abort the last command you can use `C-g'.

If you want to abort all commands, you should use the command
`abort-commands-lisp' (`C-z g').  Commands that are aborted will be put
in the buffer `*Aborted Commands*' so that you can see what was
aborted.  If you want to abort the currently running top-level command,
use `interrupt-subjob-ilisp' (`C-c C-c').  As a last resort, `M-x
panic-lisp' will reset the ILISP state without affecting the inferior
LISP so that you can see what is happening.

   `delete-char-or-pop-ilisp' (`C-d') will delete prefix characters
unless you are at the end of an ILISP buffer in which case it will pop
one level in the break loop.

   `reset-ilisp', (`C-z z') will reset the current inferior LISP's
top-level so that it will no longer be in a break loop.

\key{return-ilisp}{RET}
\key{interrupt-subjob-ilisp}{C-c C-c}
\key{abort-commands-lisp}{C-z g}
\key{M-x panic-lisp}{}
\key{reset-ilisp}{C-z z}
\key{delete-char-or-pop-ilisp}{C-d}


\subsection{Editing Lisp Forms (general)}

Indentation, parenthesis balancing, and comment commands.

\key{indent-line-ilisp}{TAB}
\key{indent-sexp-ilisp}{M-C-q}
\key{reindent-lisp}{M-q}
\key{comment-region-lisp}{C-z ;}
\key{find-unbalanced-lisp}{C-z )}
\key{close-all-lisp}{]}


\subsection{Eval and compile functions}

Functions for eval/compile in code buffers.  

\key{ilisp-prefix}{C-z} 
\key{close-and-send-lisp}{C-]}
\key{newline-and-indent-lisp}{LFD}
\key{eval-defun-lisp}{C-z e}
\key{eval-defun-lisp}{M-C-x}
\key{eval-defun-and-go-lisp}{C-z C-e} 
\key{eval-region-lisp}{C-z r}
\key{eval-region-and-go-lisp}{C-z C-r} 
\key{eval-next-sexp-lisp}{C-z n}
\key{eval-next-sexp-and-go-lisp}{C-z C-n}
\key{compile-defun-lisp}{C-z c}
\key{compile-defun-lisp-and-go}{C-z C-c}
\key{compile-region-lisp}{C-z w}
\key{compile-region-and-go-lisp}{C-z C-w}


\subsection{Documentation functions}

`describe-lisp', `inspect-lisp', `arglist-lisp', and
`documentation-lisp' switch whether they prompt for a response or use a
default when called with a negative prefix. If they are prompting,
there is completion through the inferior LISP by using `TAB' or
`M-TAB'. When entering an expression in the minibuffer, all of the
normal ilisp commands like `arglist-lisp' also work.

   Commands that work on a function will use the nearest previous
function symbol. This is either a symbol after a `\#'' or the symbol at
the start of the current list.

\key{arglist-lisp}{C-z a}
\key{documentation-lisp}{C-z d}
\key{describe-lisp}{C-z i}
\key{inspect-lisp}{C-z I}


\subsection{The Franz Online Manual}

Invoke with \kbd{M-x fi:clman} or \kbd{C-z D}, these are used to view the 
Franz online manual if available.  Packages
nicknames limit search: cltl1, comp, composer, defsys, excl, ff, inspect,
ipc, lisp, mp, prof, stream, sys, tpl, xcw, xref.  Note the useful about-*
man pages (e.g. about-top-level).

\key{fi:clman}{C-z D}
\key{fi:clman-apropos}{C-z A}
%\key{fi:clman-next-entry}{n}
%\key{fi:clman-search-forward-see-alsos}{s}
%\key{fi:clman-package-help}{p}
%\key{fi:clman-flush-doc}{C-c C-c}

\shortsubsection{Tracing functions}

Function which traces the current defun. When called with a numeric prefix the 
function will be untraced. When called with negative prefix, prompts for function to be traced.

\key{trace-defun-lisp}{C-z t}

\shortsubsection{Macroexpansion}

 These commands apply to the next sexp.  If called with a positive
 numeric prefix, the result of the macroexpansion will be inserted
 into the buffer.  With a negative prefix, prompts for expression
 to expand.

\key{macroexpand-lisp}{C-z M}
\key{macroexpand-1-lisp}{C-z m}


\shortsubsection{Package Commands}

The first time an inferior LISP mode command is executed in a Lisp
Mode buffer, the package will be determined by using the regular
expression `ilisp-package-regexp' to find a package sexp and then
passing that sexp to the inferior LISP through `ilisp-package-command'.
For the `clisp' dialect, this will find the first `(in-package
PACKAGE)' form in the file.  A buffer's package will be displayed in
the mode line.  If a buffer has no specification, forms will be
evaluated in the current inferior LISP package.

Buffer package caching can be turned off by setting the variable
`lisp-dont-cache-package' to `T'. This will force ILISP to search for
the closest previous `ilisp-package-regexp' in the buffer each time an
inferior LISP mode command is executed.

\key{package-lisp}{C-z p}
\key{set-package-lisp}{C-z P}
\key{M-x set-buffer-package-lisp}{}


\shortsubsection{Files and directories}

File commands in lisp-source-mode buffers keep track of the last used
directory and file. If the point is on a string, that will be the
default if the file exists. If the buffer is one of 
`lisp-source-modes', the buffer file will be the default. Otherwise,
the last file used in a lisp-source-mode will be used.

\key{find-file-lisp}{C-x C-f}
\key{load-file-lisp}{C-z l}
\key{compile-file-lisp}{C-z k}
\key{default-directory-lisp}{C-z !}


\shortsubsection{Completion}

Commands to reduce number of keystrokes.

\key{complete-lisp}{M-TAB}
\key{complete}{M-RET}

%\shortcopyrightnotice

\subsection{Command history}

ILISP mode is built on top of `comint-mode', the general command  
interpreter buffer mode. As such, it inherits many 
commands and features from this, including a command history mechanism.

Each ILISP buffer has a command history associated with it. Commands 
that do not match `ilisp-filter-regexp' and that are longer than 
`ilisp-filter-length' and that do not match the immediately prior 
command will be added to this history.

\key{comint-next-input}{M-n}
\key{comint-previous-input}{M-p}
\key{comint-previous-similar-input}{M-s}
\key{comint-psearch-input}{M-N}
\key{comint-msearch-input}{M-P}
\key{comint-msearch-input-matching}{C-c R}


\subsection{Source Code Commands}

   The following commands all deal with finding things in source code.
The first time that one of these commands is used, there may be some
delay while the source module is loaded.  When searching files, the
first applicable rule is used:

   * try the inferior LISP,

   * try a tags file if defined,

   * try all buffers in one of `lisp-source-modes' or all files defined
     using `lisp-directory'.

   `M-x lisp-directory' defines a set of files to be searched by the
source code commands.  It prompts for a directory and sets the source
files to be those in the directory that match entries in
`auto-mode-alist' for modes in `lisp-source-modes'.  With a positive
prefix, the files are appended.  With a negative prefix, all current
buffers that are in one of `lisp-source-modes' will be searched.  This
is also what happens by default.  Using this command stops using a tags
file.

\shortsubsection{}

`edit-definitions-lisp', `who-calls-lisp', and `edit-callers-lisp'
will switch whether they prompt for a response or use a default when
called with a negative prefix.  If they are prompting, there is
completion through the inferior LISP by using `TAB' or `M-TAB'.  When
entering an expression in the minibuffer, all of the normal ILISP
commands like `arglist-lisp' also work.

\shortsubsection{}

`edit-definitions-lisp' (`M-.') will find a particular type of
definition for a symbol.  It tries to use the rules described above.
The files to be searched are listed in the buffer `*Edit-Definitions*'.
If `lisp-edit-files' is nil, no search will be done if not found
through the inferior LISP.  The variable `ilisp-locator' contains a
function that when given the name and type should be able to find the
appropriate definition in the file.  There is often a flag to cause
your LISP to record source files that you will need to set in the
initialization file for your LISP.  The variable is
`*record-source-files*' in both allegro and lucid.  Once a definition
has been found, `next-definition-lisp' (`M-,') will find the next
definition (or the previous definition with a prefix).

\shortsubsection{}

`edit-callers-lisp' (`C-z ^') will generate a list of all of the
callers of a function in the current inferior LISP and edit the first
caller using `edit-definitions-lisp'.  Each successive call to
`next-caller-lisp' (`M-`') will edit the next caller (or the previous
caller with a prefix).  The list is stored in the buffer
`*All-Callers*'.  You can also look at the callers by doing `M-x
who-calls-lisp'.

\shortsubsection{}

`search-lisp' (`M-?') will search the current tags files,
`lisp-directory' files or buffers in one of `lisp-source-modes' for a
string or a regular expression when called with a prefix.
`next-definition-lisp' (`M-,') will find the next definition (or the
previous definition with a prefix).

\shortsubsection{}

`replace-lisp' (`M-"') will replace a string (or a regexp with a
prefix) in the current tags files, `lisp-directory' files or buffers in
one of `lisp-source-modes'.


\key{M-x lisp-directory}{}
\key{edit-definitions-lisp}{M-.}
\key{next-definition-lisp}{M-,}
\key{edit-callers-lisp}{C-z ^}
\key{next-caller-lisp}{M-`}
\key{M-x who-calls-lisp}{}
\key{search-lisp}{M-?}
\key{replace-lisp}{M-"}


\subsection{Batch commands}

The following commands all deal with making a number of changes all 
at once. The first time one of these commands is used, there may be 
some delay as the module is loaded. The eval/compile versions of these 
commands are always executed asynchronously.

 `mark-change-lisp' (`C-z SPC') marks the current defun as being
changed.  A prefix causes it to be unmarked.  `clear-changes-lisp'
(`C-z * 0') will clear all of the changes.  `list-changes-lisp' (`C-z *
l') will show the forms currently marked.

`eval-changes-lisp' (`C-z * e'), or `compile-changes-lisp' (`C-z *
c') will evaluate or compile these changes as appropriate.  If called
with a positive prefix, the changes will be kept.  If there is an
error, the process will stop and show the error and all remaining
changes will remain in the list.  All of the results will be kept in
the buffer `*Last-Changes*'.

\key{mark-change-lisp}{C-z SPC}
\key{eval-changes-lisp}{C-z * e}
\key{compile-changes-lisp}{C-z * c}
\key{clear-changes-lisp}{C-z * 0}
\key{list-changes-lisp}{C-z * l}

\subsection{Switching between interactive and raw keyboard modes}

There are two keyboard modes for interacting with the inferior LISP,
\"interactive\" and \"raw\".  Normally you are in interactive mode
where keys are interpreted as commands to EMACS and nothing is sent to
the inferior LISP unless a specific command does so.  In raw mode, all
characters are passed directly to the inferior LISP without any
interpretation as EMACS commands.  Keys will not be echoed unless
ilisp-raw-echo is T.

Raw mode can be turned on interactively by the command
`raw-keys-ilisp' (`C-z \#') and will continue until you type C-g. Raw
mode can also be turned on/off by inferior LISP functions if the
command `io-bridge-ilisp' (M-x io-bridge-ilisp) has been executed in
the inferior LISP either interactively or on a hook.  To turn on raw
mode, a function should print ^[1^] and to turn it off should print
^[0^].  An example in Common LISP would be:

\kbd{(progn (format t "1") }

\hskip 35pt \kbd{(print (read-char)) (format t "0"))}

\subsection{}

\copyrightnotice

\bye

