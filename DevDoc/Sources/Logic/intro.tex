\section{Introduction}

The \MS\ language \cite{lm} describes the construction and the manipulation of
spec(ification)s and related entities (e.g.\ morphisms). A spec is a logical
theory: it introduces symbols and asserts properties of those symbols; further
properties are derived via logical inference.

This document formalizes the logic of the \MS\ language, i.e.\ which judgements
can be asserted and how judgements are inferred from other judgements. This
document does not describe the full semantics of the \MS\ language, but only its
underlying logic. The full \MS\ semantics is formalized in \cite{semantics}:
that document is based on the formalization of the logic in this document. The
\MS\ logic is defined on a relatively small ``core'' subset of \MS; more complex
\MS\ constructs are defined in terms of the core construct \cite{semantics}.

The \MS\ logic consists of standard higher-order logic \cite{andrews} plus
record types, polymorphic types, predicate subtypes, and an if-then-else. Record
types are straightfoward. Polymorphic types are as in the HOL system \cite{hol}
and are not particularly difficult. Predicates subtypes are as in the PVS system
\cite{pvs-seman}: they make well-typedness undecidable because in order to
establish whether an expression of a supertype has also a subtype it is
necessary to prove that the predicate that defines the subtype is true for the
expression, in the context where the expression occurs. Thus, inference rules to
derive theorems and inference rules to derive well-typedness judgements are
mutually recursive, together with the rules to derive other kinds of judgements.
Unlike an ordinary ternary function, the contexts to establish the
well-typedness of the second and third argument of an if-then-else are extended
with the truth and falsehood of the first argument; this is why if-then-else is
a primitive construct in the \MS\ logic.

The rest of this section introduces the mathematical notation used in this
document. \secref{syntax} defines the core (abstract) syntax of the \MS\ logic
while \secref{proofth} defines its proof theory. \secref{props} states
properties of the syntax and proof theory. \secref{models} defines a
set-theoretic semantics of the \MS\ logic. Finally, \secref{proofs} collects all
the (meta) proofs of the (meta) theorems stated in the other sections.

\subsection{Notation}

We define the \MS\ logic in the usual semi-formal notation consisting of naive
set theory and natural language. However, it is possible to define the \MS\
logic in axiomatic set theory or any other sufficiently expressive formal
language.\footnote{For example, Peter Homeier has formalized the \MS\ logic in
the logic of the HOL theorem prover (see Acknowledgments section for details).}

The (meta) logical notations $=$, $\forall$, $\exists$, $\neg$, $\wedge$,
$\vee$, $\Rightarrow$, $\Leftrightarrow$, and $\ \not\!\!\!\!\ldots$ (e.g.\
$\neq$) have the usual meaning.

The set-theoretic notations $\in$, $\emptyset$, $\setST{\ldots}{\ldots}$,
$\setI{\ldots}$, $\cup$, $\cap$, and $\subseteq$ have the usual meaning.

$\N$ is the set of natural numbers, i.e.\ $\setIV{0}{1}{2}{\ldots}$.

%$\NP$ is the set of positive natural numbers, i.e.\ $\setIV{1}{2}{3}{\ldots}$.

If $A$ and $B$ are sets, $A-B$ is their difference, i.e.\ $\setST{x\in
A}{x\not\in B}$.

If $A$ and $B$ are sets, $A\times B$ is their cartesian product, i.e.\
$\setST{\tupII{a}{b}}{a\in A\AND b\in B}$. This generalizes to $n>2$ sets.

If $A$ and $B$ are sets, $A+B$ is their disjoint union, i.e.\
$\setST{\tupII{0}{a}}{a\in A}\cup\setST{\tupII{1}{b}}{b\in B}$. The ``tags'' 0
and 1 are always left implicit. This generalizes to $n>2$ sets.

If $A$ and $B$ are sets: $\Funcp{A}{B}$ is the set of all partial functions from
$A$ to $B$, i.e.\ $\setST{f\subseteq A\times
B}{\FORALL{\tupII{a}{b_1},\tupII{a}{b_2}\in f}{b_1=b_2}}$; $\Func{A}{B}$ is the
set of all total functions from $A$ to $B$, i.e.\
$\setST{f\in\Funcp{A}{B}}{\FORALL{a\in A}{\EXISTS{b\in B}{\tupII{a}{b}\in f}}}$;
%$\Funcf{A}{B}$ is the set of all finite functions from $A$ to $B$, i.e.\
%$\setST{f\in\Funcp{A}{B}}{f\mbox{ is a finite set}}$;
and $\Funcinj{A}{B}$ is the set of all total injective functions from $A$ to
$B$, i.e.\ $\setST{f\in\Func{A}{B}}{\FORALL{\tupII{a_1}{b},\tupII{a_2}{b}\in
f}{a_1=a_2}}$.  We write $\funcp{f}{A}{B}$, $\func{f}{A}{B}$,
%$\funcf{f}{A}{B}$,
and $\funcinj{f}{A}{B}$ for $f\in\Funcp{A}{B}$, $f\in\Func{A}{B}$,
%$f\in\Funcf{A}{B}$,
and $f\in\Funcinj{A}{B}$, respectively.

If $f$ is a function from $A$ to $B$: $\dom{f}$ is the domain of $f$, i.e.\
$\setST{a\in A}{\EXISTS{b\in B}{\tupII{a}{b}\in f}}$; $\rng{f}$ is the range of
$f$, i.e.\ $\setST{b\in B}{\EXISTS{a\in A}{\tupII{a}{b}\in f}}$.

If $f$ is a function and $a\in\dom{f}$, $f(a)$ denotes the unique value such
that $\tupII{a}{f(a)}\in f$.

If $f$ is a function and $A$ is a set: $\funrestr{f}{A}$ denotes the restriction
of $f$ to $A$, i.e.\ $\setST{\tupII{a}{f(a)}}{a\in\dom{f}\cap A}$ (we may have
$A\subseteq\dom{f}$ or not); $\funrestrc{f}{A}$ denotes the restriction of $f$
to $\dom{f}-A$, i.e.\ $\funrestr{f}{\dom{f}-A}$.

If $A$ is a set, $\Setf{A}$ is the set of all finite subsets of $A$, i.e.\
$\setST{S\subseteq A}{S\mbox{ finite}}$.

If $A$ is a set, $\Seq{A}$ is the set of all finite sequences of elements of
$A$, i.e.\ $\setST{\seqFT{x_1}{x_n}}{x_1\in A\AND\ldots\AND x_n\in A}$;
$\SeqNE{A}$, $\SeqNR{A}$, and $\SeqNER{A}$ are the subsets of $\Seq{A}$ of
non-empty sequences, sequences without repeated elements, and non-empty
sequences without repeated elements, respectively.\footnote{Strictly speaking,
our notation $\seqFT{x_1}{x_n}$ for sequences may lead to ambiguities, e.g.\ if
$s_1$ and $s_2$ are sequences, is $s_1,s_2$ the sequence of length 2 whose
elements are $s_1$ and $s_2$, or is it the concatenation of $s_1$ and $s_2$?
However, in this document, the intended meaning should be always clear from the
symbols used and from mathematical context.} The empty sequence is written
$\seqO$. A sequence $\seqFT{x_1}{x_n}$ is often written $\seq{x}$, leaving $n$
implicit. The length of a sequence $s$ is written $\seqlen{s}$.  When a sequence
is written where a set is expected, it stands for the set of its elements.
