\section{Models}
\label{models}

[[[TO DO]]]

This section defines the notion of model of a context.

A model of a context is a mapping from names declared in the context to suitable
set-theoretic entities. For instance, a type name $\tnam$ of arity $n$ is mapped
to an $n$-ary function over sets (if $n=0$, the model maps the type name simply
to a set). The mapping is extended to all well-formed types, which are mapped to
sets, and to all well-typed expressions, which are mapped to elements of the
sets that their types map to. The model must satisfy all the axioms of the
context.

It should be possible to prove the soundness of the rules to derive judgements
with respect to models.

Since higher-order logic is notoriously incomplete, it is not possible to prove
completeness of the rules to derive assertions. However, it should be possible
to prove completeness with respect to general (a.k.a.\ Henkin) models. A general
model is one in which the type $\tarrO$ is a subset of all functions from
$\typ_1$ to $\typ_2$, and not necessarily the set of all such functions (as in
standard models). Since there are more general models than standard models (a
standard model is also a general model but not all general models are standard
models), fewer formulas are true in all general models than in all standard
models.

Perhaps this section should also contain a proof of the consistency of the \MS\
logic, analogously to the proof of the consistency of the higher-order logic
defined in \cite{andrews}.
