\section{Syntax}
\label{syntax}

\subsection{Names}

We postulate the existence of an infinite set of names
\[
\Nam
\]

%We assume that $\Nam$ contains a distinguished name $\projnam{i}$ for every
%positive natural $i\in\NP$, such that $\projnam{i}\neq\projnam{j}$ if $i\neq
%j$ and such that $\Nam-\setST{\projnam{i}}{i\in\NP}$ is still infinite.

\subsection{Types}

We inductively define the set of types as
\[
\setdefVIIIoI
 {\Typ}
 {+}
 {\setI{\bool}}
 {\setST{\tvar}{\tvar\in\Nam}}
 {\setST{\tinstO}{\tnam\in\Nam\AND\typS\in\Seq{\Typ}}}
 {\setST{\tarr{\typ_1}{\typ_2}}{\typ_1,\typ_2\in\Typ}}
 {\setST{\trecFTO}{\fnamS\in\SeqNR{\Nam}\AND\typS\in\Seq{\Typ}}}
 {\setST{\tsumFTO}{\cnamS\in\SeqNER{\Nam}\AND\typS\in\SeqNE{\Typ}}}
 {\setST{\tsubO}{\typ\in\Typ\AND\tspred\in\Expr}}
\setdefVIIIoII
 {+}
 {\setST{\tquotO}{\typ\in\Typ\AND\tqpred\in\Expr}}
\]
where $\Expr$ is defined later.\footnote{Types depend on expressions, which
depend on types. Thus, types and expressions are inductively defined together,
not separately. Their definitions are presented separately only for
readability.}

Explanation:
\begin{itemize}
\item
There is a type $\bool$ for boolean (i.e.\ truth) values.
\item
A name $\tvar$ is a type variable.
\item
A type instance $\tinstO$ is obtained by combining a type name $\tnam$ with
zero or more argument types $\typS$. We may write $\tinst{\tnam}{\seqO}$ as
just $\tnam$.
\item
An arrow type $\tarrO$ consists of a domain $\typ_1$ and a range $\typ_2$.
\item
Record types $\trecO$ (resp.\ sum types $\tsumO$) consist of typed fields
$\fnam_i$ (resp.\ constructors $\cnam_i$). There exists one record type that
has no fields (i.e.\ $n=0$), denoted $\trecmt$, whose only inhabitant is the
empty record. A sum type, instead, always has at least one constructor. All
the fields (resp.\ constructors) of a record (resp.\ sum) type must be
distinct names. The case of no type $\typ_i$ associated to a constructor
$\cnam_i$ in a sum type as defined in \cite{lm} is captured by $\typ_i$ being
$\trecmt$ in the definition above: given a spec as defined in \cite{lm}, one
can imagine to add $\trecmt$ where a constructor has no type, and add the
empty record as argument to $\cnam_i$ in expressions and patterns where
needed.
\item
Restriction types $\tsubO$ and quotient types $\tquotO$ are obtained by
combining types $\typ$ with expressions $\tspred$ and $\tqpred$ (meant to be
suitable predicates). Restriction and quotient types create the dependency of
types on expressions. Restriction types as defined above also capture
comprehension types as defined in \cite{lm}: as mentioned in \cite{lm}, a
comprehension type can be turned into a restriction type by re-combining the
pattern and expression into a lambda expression.
\end{itemize}

We introduce the abbreviation
\[
\begin{abblist}
\abbitem{\tprodO}
        {\trec{i}{\projnam{i}}{\typ_i}}
\end{abblist}
\]
where each $\projnam{i}$ is a fixed but unspecified name in $\Nam$ such that
$\projnam{i}\neq\projnam{j}$ if $i\neq j$. Thus, record types as defined above
also capture product types as defined in \cite{lm}; the names $\projnam{i}$
capture the natural literal fields used in \cite{lm}.

\subsection{Expressions}

We inductively define the set of expressions as
\[
\setdefXoI
 {\Expr}
 {+}
 {\setST{\var}{\var\in\Nam}}
 {\setST{\opO}{\onam\in\Nam\AND\typS\in\Seq{\Typ}}}
 {\setST{\appO}{\expr_1,\expr_2\in\Expr}}
 {\setST{\absO}{\var\in\Nam\AND\typ\in\Typ\AND\expr\in\Expr}}
 {\setST{\eqO}{\expr_1,\expr_2\in\Expr}}
% {\setST{\iifopO}{\typ\in\Typ}}
 {\setST{\iifO}{\expr_0,\expr_1,\expr_2\in\Expr}}
 {\setST{\descopO}{\typ\in\Typ}}
% {\setST{\descO}{\var\in\Nam\AND\typ\in\Typ\AND\expr\in\Expr}}
\setdefXoII
 {+}
 {\setST{\projopO}{\trecO\in\Typ}}
% {\setST{\projO}{\expr\in\Expr\AND\fnam\in\Nam}}
 {\setST{\embedO}{\tsumO\in\Typ}}
 {\setST{\quotO}{\tquotO\in\Typ}}
\]

Explanation:
\begin{itemize}
\item
A name $\var$ is a variable.
\item
An op(eration) instance $\opO$ consists of an op name $\onam$ and zero or more
types $\typS$ that instantiate the (generally, polymorphic) type of the
declared op. We may write $\op{\onam}{\seqO}$ as just $\onam$.
\item
An application $\appO$ consists of a function $\expr_1$ juxtaposed to an
argument $\expr_2$.
\item
A (lambda) abstraction $\absO$ consists of an argument $\var$ with an explicit
type $\typ$ and a body $\expr$. Even though lambda expressions as defined in
\cite{lm} may have branches with patterns, that does not increase
expressivity: one can imagine to use a fresh variable as argument of the
abstraction and a case expression (introduced later) on the fresh variable
with the branches as the body.
\item
An equality $\eqO$ consists of a left-hand side $\expr_1$ and a right-hand
side $\expr_2$.
\item
%The conditional operator $\iifopO$ is decorated by a type. It captures
%if-then-else as defined in \cite{lm}: it operates on a boolean value and two
%values of type $\typ$, selectively returning one of them. The conditional
%operator is non-strict. We may write
%$\app{\app{\app{\iifopO}{\expr_0}}{\expr_1}}{\expr_2}$ as just $\iifO$ when
%the type is inferrable or irrelevant.
A conditional $\iifO$ consists of a condition $\expr_0$, a ``then'' branch
$\expr_1$, and an ``else'' branch $\expr_2$.
\item
The description operator $\descopO$ is tagged by a type. It operates on
predicates over $\typ$ (i.e.\ over values of type $\tarr{\typ}{\bool}$) that
are satisfied by a unique value of $\typ$, and its result is \emph{the} value
that satisfies the predicate.
%A description $\descO$, currently absent from \cite{lm}, consists of a typed
%variable and a body (like an abstraction). Its informal meaning is the
%following: $\expr$ must have type $\bool$ and be satisfied by exactly one
%value of type $\typ$ assigned to $\var$; the result of $\descO$ is that unique
%value. In other words, $\descO$ is the $\var$ of type $\typ$ that satisfies
%$\expr$ (we say ``the'' because the value exists and is unique).
\item
A projector $\projopO$ is tagged by a record type and by a field of that
record type. We may write $\projopO$ as just $\pjopO$ when the record type is
inferrable or irrelevant.
%A projection $\projO$ consists of an expression and a field.
\item
An embedder $\embedO$ is tagged by a sum type and by a constructor of that sum
type. We may write $\embedO$ as just $\embO$ when the sum type is inferrable
or irrelevant.
\item
A quotienter $\quotO$ is tagged by a quotient type. We may write $\quotO$ as
just $\quo$ when the quotient type is inferrable or irrelevant.
%\item
%Instead of introducing equalities $\eqO$, we could have introduced an equality
%operator $\eqopO$ to be applied to expressions $\expr_1$ and $\expr_2$
%(analogously to the abbreviation $\iiffop$, introduced later). However, the
%type $\typ$ decorating $\eqopO$ (necessary, otherwise the type of $\eqY$ would
%be unclear) would make it necessary to introduce inference rules that relate
%equalities of different types, e.g.\ a rule stating that $\eqtO$ implies
%$\eqt{\typ'}{\expr_1}{\expr_2}$ if $\typ$ is a subtype of $\typ'$. We prefer
%to avoid these additional rules and use undecorated, ``applied'' equalities.
\end{itemize}

We introduce the abbreviations (many of which are expressions defined in
\cite{lm})
\[
\begin{abblist}
\abbitem{\true}
        {\eq{\abs{\varfx}{\bool}{\varfx}}{\abs{\varfx}{\bool}{\varfx}}}
\abbitem{\false}
        {\eq{\abs{\varfx}{\bool}{\varfx}}{\abs{\varfx}{\bool}{\true}}}
\abbitem{\negaop}
        {\abs{\varfx}{\bool}{(\iif{\varfx}{\false}{\true})}}
%\abbitem{\negaO}
%        {\iif{\expr}{\false}{\true}}
\abbitem{\conjO}
        {\iif{\expr_1}{\expr_2}{\false}}
\abbitem{\disjO}
        {\iif{\expr_1}{\true}{\expr_2}}
\abbitem{\implO}
        {\iif{\expr_1}{\expr_2}{\true}}
\abbitem{\iiffop}
        {\abs{\varfx}{\bool}{\abs{\varfx'}{\bool}{(\eq{\varfx}{\varfx'})}}}
\abbitem{\iiffO}
        {\app{\app{\iiffop}{\expr_1}}{\expr_2}}
%\abbiteM{\iiffO}
%        {\eqO}
%        {\mbox{$\expr_1$ and $\expr_2$ have type $\bool$}}
\abbitem{\neeqO}
        {\nega{(\eqO)}}
\abbitem{\descO}
        {\app{\descopO}{(\absO)}}
\abbitem{\faopO}
        {\abs{\varfxpred}{\tarr{\typ}{\bool}}
             {(\eq{\varfxpred}{\abs{\varfx}{\typ}{\true}})}}
\abbitem{\faO}
        {\app{\faopO}{(\absO)}}
%        {\eq{\absO}{\abs{\var}{\typ}{\true}}}
\abbitem{\faS{\seqFT{\bnd{\var_1}{\typ_1}}{\bnd{\var_n}{\typ_n}}}{\expr}}
        {\fa{\var_1}{\typ_1}{\ldots\fa{\var_n}{\typ_n}{\expr}}}
\abbitem{\fa{\varS}{\typS}{\expr}}
        {\faS{\seqFT{\bnd{\var_1}{\typ_1}}{\bnd{\var_n}{\typ_n}}}{\expr}}
\abbitem{\exopO}
        {\abs{\varfxpred}{\tarr{\typ}{\bool}}
             {\nega{(\fa{\varfx}{\typ}{\nega{(\app{\varfxpred}{\varfx})}})}}}
\abbitem{\exO}
        {\app{\exopO}{(\absO)}}
%        {\nega{(\fa{\var}{\typ}{\nega{\expr}})}}
\abbitem{\exS{\seqFT{\bnd{\var_1}{\typ_1}}{\bnd{\var_n}{\typ_n}}}{\expr}}
        {\ex{\var_1}{\typ_1}{\ldots\ex{\var_n}{\typ_n}{\expr}}}
\abbitem{\ex{\varS}{\typS}{\expr}}
        {\exS{\seqFT{\bnd{\var_1}{\typ_1}}{\bnd{\var_n}{\typ_n}}}{\expr}}
\abbitem{\exIopO}
        {\abs{\varfxpred}{\tarr{\typ}{\bool}}
             {(\ex{\varfx}{\typ}
                  {(\conj{\app{\varfxpred}{\varfx}}
                         {\fa{\varfx'}{\typ}
                             {(\impl{\app{\varfxpred}{\varfx'}}
                                    {\eq{\varfx'}{\varfx}})}})})}}
\abbitem{\exIO}
        {\app{\exIopO}{(\absO)}}
%        {\ex{\var}
%            {\typ}
%            {(\conj{\expr}
%                   {\fa{\var'}{\typ}
%                       {\impl{\esbs{\expr}{\var}{\var'}}{\eq{\var}{\var'}}}})}}
%        {\var'\neq\var\AND
%         \var'\not\in\efvarO\AND
%         \var'\not\in\cvarv{\expr}{\var}}
%\abbitem{\absS{\seqFT{\bnd{\var_1}{\typ_1}}{\bnd{\var_n}{\typ_n}}}{\expr}}
%        {\abs{\var_1}{\typ_1}
%             {\ldots\abs{\var_n}{\typ_n}{\expr}}}
%\abbitem{\abs{\varS}{\typS}{\expr}}
%        {\absS{\seqFT{\bnd{\var_1}{\typ_1}}{\bnd{\var_n}{\typ_n}}}{\expr}}
\abbitem{\projO}
        {\app{\pjop{\fnam}}{\expr}}
\abbitem{\recopO}
        {\abs{\varfxI{1}}{\typ_1}
             {\ldots\abs{\varfxI{n}}{\typ_n}
                        {\desc{\varfx}{\trecO}
                              {\conjI{i}{(\eq{\proj{\varfx}{\fnam_i}}
                                             {\varfxI{i}})}}}}}
\abbitem{\rectFTO}
        {\app{\app{\recopO}{\expr_1}}{\ldots\;\expr_n}}
\abbitem{\tuple{\seqFT{\expr_1}{\expr_n}}}
        {\recFT{\projnam{1}}{\expr_1}{\projnam{n}}{\expr_n}}
\end{abblist}
\]
where $\varfx$, $\varfx'$, $\varfxpred$, and each $\varfxI{i}$ are fixed but
unspecified names in $\Nam$ such that they are all distinct.

Explanation:
\begin{itemize}
\item
The abbreviations $\true$ and $\false$ stand for logical truth and falsehood,
respectively.
%It is intended that infinitely many expressions are abbreviated by $\true$ or
%$\false$, one for each $\var\in\Nam$. Note that $\abs{\var}{\bool}{\var}$ is
%the simplest expression that is well-typed in every context, including the
%empty context (contexts and well-typedness are defined later).
\item
The logical connectives for negation, conjunction, disjunction, and
implication are defined in terms of conditionals.
\item
Coimplication is a synonym for equality, but only for booleans.
\item
Inequality is negation of equality.
\item
Stand-alone quantifiers are higher-order functions, but they can be written in
binder form. Also the description operator can be written in binder form. Note
that both $\faS{\seqO}{\expr}$ and $\exS{\seqO}{\expr}$ stand for $\expr$.
\item
A dotted projection $\projO$ abbreviates an applied projection whose record
type is implicit; this is why no record type appears in $\projO$.
\item
A record constructor $\recopO$ is tagged by a record type and is defined by
means of a description; records are characterized by their components. We may
write $\recopO$ as just $\rcop$ when the record type is inferrable or
irrelevant. Note that $\recop{\trecmt}$ denotes the empty record (a
conjunction $\conjI{i}{\expr_i}$ consisting of no conjuncts, i.e.\ such that
$n=0$, stands for $\true$).  The notation $\rectO$ is a more readable version
of an applied record constructor. We may write $\rectO$ as just $\recO$ when
the component types are inferrable or irrelevant.
%A record $\recO$ abbreviates infinitely many expressions as well, one for each
%$\var\in\Nam$ and $\typS\in\Seq{\Typ}$ satisfying the given restrictions (as
%defined later, an expression may have more than one type). This abbreviation
%makes use of a description because a record is characterized by its
%components.
\item
A tuple $\tupleO$ captures tuple displays as defined in \cite{lm}. It
abbreviates a record construction of a product type whose component types are
implicit; this is why no component types appear in $\tupleO$.
\end{itemize}

%With the exception of unique existentials, all the above abbreviations
%correspond to expressions defined in \cite{lm}. As shown later, all the other
%constructs defined in \cite{lm}, including patterns, are captured as
%abbreviations of the expressions introduced above.

The function $\func{\efvarY}{\Expr}{\Setf{\Nam}}$ returns the free variables
of an expression
\[
\begin{eqlist}
\eqitem{\efvar{\var}}        {\setI{\var}}
\eqitem{\efvar{\opO}}        {\emptyset}
\eqitem{\efvar{\appO}}       {\efvar{\expr_1}\cup\efvar{\expr_2}}
\eqitem{\efvar{\absO}}       {\efvar{\expr}-\setI{\var}}
\eqitem{\efvar{\eqO}}        {\efvar{\expr_1}\cup\efvar{\expr_2}}
\eqitem{\efvar{\iifO}}       {\efvar{\expr_0}\cup
                              \efvar{\expr_1}\cup\efvar{\expr_2}}
\eqitem{\efvar{\descopO}}    {\emptyset}
\eqitem{\efvar{\pjop{\fnam}}}{\emptyset}
\eqitem{\efvar{\emb{\cnam}}} {\emptyset}
\eqitem{\efvar{\quo}}        {\emptyset}
\end{eqlist}
\]
Note that we do not consider the free variables in expressions contained in
types contained in expressions (e.g.\ we do not consider the free variables in
$\tqpred$ as part of the free variables of $\quotO$). The reason is that, as
defined later, all the expressions contained in well-formed types have no free
variables.
%The free variables of the quotient type predicate $\tqpred$ do not contribute
%to the free variables of $\quotO$ because, as defined later, in well-formed
%types and well-typed expressions those predicates have no free variables.

\subsection{Contexts}

We define the set of context elements as
\[
\setdefVII
 {\Cxel}
 {+}
 {\setST{\tdecO}{\tnam\in\Nam\AND n\in\N}}
 {\setST{\odecO}{\onam\in\Nam\AND\tvarS\in\SeqNR{\Nam}\AND\typ\in\Typ}}
 {\setST{\tdefO}{\tnam\in\Nam\AND\tvarS\in\SeqNR{\Nam}\AND\typ\in\Typ}}
% {\setST{\odefO}{\tvarS\in\SeqNR{\Nam}\AND\onam\in\Nam\AND\expr\in\Expr}}
 {\setST{\axO}{\tvarS\in\SeqNR{\Nam}\AND\expr\in\Expr}}
 {\setST{\lemO}{\tvarS\in\SeqNR{\Nam}\AND\expr\in\Expr}}
 {\setST{\tvdecO}{\tvar\in\Nam}}
 {\setST{\vdecO}{\var\in\Nam\AND\typ\in\Typ}}
\]

Explanation:
\begin{itemize}
\item
A type declaration $\tdecO$ introduces a type name with an associated arity.
The type variables of a type declaration as defined in \cite{lm} only serve to
determine an arity and are otherwise irrelevant; thus, in the above definition
we directly use the arity without type variables.
\item
An op(eration) declaration $\odecO$ introduces an op name with an associated
type, polymorphic in the explicit type variables.
\item
A type definition $\tdefO$ assigns a type to a maximally generic type instance
of some type name (i.e.\ an instance with distinct type variables as arguments
to the type name). A combined type declaration and definition as defined in
\cite{lm} is captured by a type declaration as defined above immediately
followed by a type definition as defined above.
%\item
%An op definition $\odefO$ assigns an expression to an op name, polymorphic in
%the explicit type variables. A combined op declaration and definition as
%defined in \cite{lm} is captured by an op declaration as defined above
%immediately followed by an op definition as defined above.
\item
An axiom $\axO$ introduces an expression (with type $\bool$, as defined
later), polymorphic in the explicit type variables. We may write
$\ax{\seqO}{\expr}$ as just $\axM{\expr}$.
\item
A lemma $\lemO$ captures a theorem or conjecture as defined in \cite{lm}. We
may write $\lem{\seqO}{\expr}$ as just $\lemM{\expr}$.
\item
A type variable declaration $\tvdecO$ introduces a type variable. We may write
$\seqFT{\tvdec{\tvar_1}}{\tvdec{\tvar_n}}$ as just
$\tvdec{\seqFT{\tvar_1}{\tvar_n}}$.
\item
A variable declaration $\vdecO$ introduces a variable with a type.
\end{itemize}

We define the set of contexts as
\[
\Cx = \Seq{\Cxel}
\]
In other words, a context is a finite sequence of context elements.

The function $\func{\cxtnamY}{\Cx}{\Setf{\Nam}}$ returns the type names
declared in a context
\[
\begin{eqlist}
\eqitem{\cxtnam{\mtcx}}
       {\emptyset}
\eqitem{\cxtnam{\cons{\cxel}{\cx}}}
       {\cond{\cxel=\tdecO}
             {\cxtnam{\cx}\cup\setI{\tnam}}
             {\cxtnam{\cx}}}
%\eqitem{\cxtnam{\cons{\tdecO}{\cx}}} {\cxtnam{\cx}\cup\setI{\tnam}}
%\eqitem{\cxtnam{\cons{\odecO}{\cx}}} {\cxtnam{\cx}}
%\eqitem{\cxtnam{\cons{\tdefO}{\cx}}} {\cxtnam{\cx}}
%\eqitem{\cxtnam{\cons{\odefO}{\cx}}} {\cxtnam{\cx}}
%\eqitem{\cxtnam{\cons{\axO}{\cx}}}   {\cxtnam{\cx}}
%\eqitem{\cxtnam{\cons{\tvdecO}{\cx}}}{\cxtnam{\cx}}
%\eqitem{\cxtnam{\cons{\vdecO}{\cx}}} {\cxtnam{\cx}}
\end{eqlist}
\]

The function $\func{\cxonamY}{\Cx}{\Setf{\Nam}}$ returns the op names declared
in a context
\[
\begin{eqlist}
\eqitem{\cxonam{\mtcx}}
       {\emptyset}
\eqitem{\cxonam{\cons{\cxel}{\cx}}}
       {\cond{\cxel=\odecO}
             {\cxonam{\cx}\cup\setI{\onam}}
             {\cxonam{\cx}}}
%\eqitem{\cxonam{\cons{\tdecO}{\cx}}} {\cxonam{\cx}}
%\eqitem{\cxonam{\cons{\odecO}{\cx}}} {\cxonam{\cx}\cup\setI{\onam}}
%\eqitem{\cxonam{\cons{\tdefO}{\cx}}} {\cxonam{\cx}}
%\eqitem{\cxonam{\cons{\odefO}{\cx}}} {\cxonam{\cx}}
%\eqitem{\cxonam{\cons{\axO}{\cx}}}   {\cxonam{\cx}}
%\eqitem{\cxonam{\cons{\tvdecO}{\cx}}}{\cxonam{\cx}}
%\eqitem{\cxonam{\cons{\vdecO}{\cx}}} {\cxonam{\cx}}
\end{eqlist}
\]

The function $\func{\cxtvarY}{\Cx}{\Setf{\Nam}}$ returns the type variables
declared in a context
\[
\begin{eqlist}
\eqitem{\cxtvar{\mtcx}}
       {\emptyset}
\eqitem{\cxtvar{\cons{\cxel}{\cx}}}
       {\cond{\cxel=\tvdecO}
             {\cxtvar{\cx}\cup\setI{\tvar}}
             {\cxtvar{\cx}}}
%\eqitem{\cxtvar{\cons{\tdecO}{\cx}}} {\cxtvar{\cx}}
%\eqitem{\cxtvar{\cons{\odecO}{\cx}}} {\cxtvar{\cx}}
%\eqitem{\cxtvar{\cons{\tdefO}{\cx}}} {\cxtvar{\cx}}
%\eqitem{\cxtvar{\cons{\odefO}{\cx}}} {\cxtvar{\cx}}
%\eqitem{\cxtvar{\cons{\axO}{\cx}}}   {\cxtvar{\cx}}
%\eqitem{\cxtvar{\cons{\tvdecO}{\cx}}}{\cxtvar{\cx}\cup\setI{\tvar}}
%\eqitem{\cxtvar{\cons{\vdecO}{\cx}}} {\cxtvar{\cx}}
\end{eqlist}
\]

The function $\func{\cxvarY}{\Cx}{\Setf{\Nam}}$ returns the variables declared
in a context
\[
\begin{eqlist}
\eqitem{\cxvar{\mtcx}}
       {\emptyset}
\eqitem{\cxvar{\cons{\cxel}{\cx}}}
       {\cond{\cxel=\vdecO}
             {\cxvar{\cx}\cup\setI{\var}}
             {\cxvar{\cx}}}
%\eqitem{\cxvar{\cons{\tdecO}{\cx}}} {\cxvar{\cx}}
%\eqitem{\cxvar{\cons{\odecO}{\cx}}} {\cxvar{\cx}}
%\eqitem{\cxvar{\cons{\tdefO}{\cx}}} {\cxvar{\cx}}
%\eqitem{\cxvar{\cons{\odefO}{\cx}}} {\cxvar{\cx}}
%\eqitem{\cxvar{\cons{\axO}{\cx}}}   {\cxvar{\cx}}
%\eqitem{\cxvar{\cons{\tvdecO}{\cx}}}{\cxvar{\cx}}
%\eqitem{\cxvar{\cons{\vdecO}{\cx}}} {\cxvar{\cx}\cup\setI{\var}}
\end{eqlist}
\]

\subsection{Specs}

We define the set of spec(ification)s as
\[
\Spec = \setST{\cx\in\Cx}{\cxtvar{\cx}=\cxvar{\cx}=\emptyset}
\]
In other words, a spec is a context without type variable declarations and
variable declarations.

%\subsection{Occurring ops}

%The function $\func{\opsinY}{\Typ+\Expr}{\Setf{\Nam}}$ returns the op names
%occurring in a type or expression
%\[
%\begin{eqlist}
%\eqitem{\opsin{\bool}}  {\emptyset}
%\eqitem{\opsin{\tvar}}  {\emptyset}
%\eqitem{\opsin{\tinstO}}{\bigcup_i\opsin{\typ_i}}
%\eqitem{\opsin{\tarrO}} {\opsin{\typ_1}\cup\opsin{\typ_2}}
%\eqitem{\opsin{\trecO}} {\bigcup_i\opsin{\typ_i}}
%\eqitem{\opsin{\tsumO}} {\bigcup_i\opsin{\typ_i}}
%\eqitem{\opsin{\tsubO}} {\opsin{\typ}\cup\opsin{\tspred}}
%\eqitem{\opsin{\tquotO}}{\opsin{\typ}\cup\opsin{\tqpred}}
%\end{eqlist}
%\]
%\[
%\begin{eqlist}
%\eqitem{\opsin{\var}}    {\emptyset}
%\eqitem{\opsin{\opO}}    {\setI{\onam}\cup\bigcup_i\opsin{\typ_i}}
%\eqitem{\opsin{\appO}}   {\opsin{\expr_1}\cup\opsin{\expr_2}}
%\eqitem{\opsin{\absO}}   {\opsin{\typ}\cup\opsin{\expr}}
%\eqitem{\opsin{\eqO}}    {\opsin{\expr_1}\cup\opsin{\expr_2}}
%\eqitem{\opsin{\iifO}}   {\opsin{\expr_0}\cup\opsin{\expr_1}\cup\opsin{\expr_2}}
%\eqitem{\opsin{\descopO}}{\opsin{\typ}}
%\eqitem{\opsin{\projopO}}{\opsin{\trecO}}
%\eqitem{\opsin{\embedO}} {\opsin{\tsumO}}
%\eqitem{\opsin{\quotO}}  {\opsin{\tquotO}}
%\end{eqlist}
%\]

\subsection{Substitutions}

\subsubsection{Type substitutions}

The function
$\func{\tsbsY}{(\Typ+\Expr)\times(\Funcp{\Nam}{\Typ})}{\Typ+\Expr}$
substitutes each type variable $\tvar\in\dom{\tsbsmap}$, where
$\funcp{\tsbsmap}{\Nam}{\Typ}$, with the type $\tsbsmap(\tvar)$ in a type or
expression $x$ (written $\tsbsO{x}$)
\[
\begin{eqlistC}
\eqitem{\tsbsO{\bool}}
       {\bool}
\eqitem{\tsbsO{\tvar}}
       {\cond{\tvar\in\dom{\tsbsmap}}{\tsbsmap(\tvar)}{\tvar}}
\eqitem{\tsbsO{\tinstO}}
       {\tinst{\tnam}{\tsbsO{\typS}}}
\eqitem{\tsbsO{(\tarrO)}}
       {\tarr{\tsbsO{\typ_1}}{\tsbsO{\typ_2}}}
\eqitem{\tsbsO{(\trecO)}}
       {\trec{i}{\fnam_i}{\tsbsO{\typ_i}}}
\eqitem{\tsbsO{(\tsumO)}}
       {\tsum{i}{\cnam_i}{\tsbsO{\typ_i}}}
\eqitem{\tsbsO{(\tsubO)}}
       {\tsub{\tsbsO{\typ}}{\tsbsO{\tspred}}}
\eqitem{\tsbsO{(\tquotO)}}
       {\tquot{\tsbsO{\typ}}{\tsbsO{\tqpred}}}
\end{eqlistC}
\]
\[
\begin{eqlistC}
\eqitem{\tsbsO{\var}}
       {\var}
\eqitem{\tsbsO{\opO}}
       {\op{\onam}{\tsbsO{\typS}}}
\eqitem{\tsbsO{(\appO)}}
       {\app{\tsbsO{\expr_1}}{\tsbsO{\expr_2}}}
\eqitem{\tsbsO{(\absO)}}
       {\abs{\var}{\tsbsO{\typ}}{\tsbsO{\expr}}}
\eqitem{\tsbsO{(\eqO)}}
       {\eq{\tsbsO{\expr_1}}{\tsbsO{\expr_2}}}
\eqitem{\tsbsO{(\iifO)}}
       {\iif{\tsbsO{\expr_0}}{\tsbsO{\expr_1}}{\tsbsO{\expr_2}}}
\eqitem{\tsbsO{(\descopO)}}
       {\descop{\tsbsO{\typ}}}
\eqitem{\tsbsO{\left(\projopO\right)}}
       {\projop{\tsbsO{\left(\trecO\right)}}{\fnam_j}}
\eqitem{\tsbsO{\left(\embedO\right)}}
       {\embed{\tsbsO{\left(\tsumO\right)}}{\cnam_j}}
\eqitem{\tsbsO{\left(\quotO\right)}}
       {\quot{\tsbsO{(\tquotO)}}}
\end{eqlistC}
\]
where of course
$\tsbsO{(\seqFT{\typ_1}{\typ_n})}=\seqFT{\tsbsO{\typ_1}}{\tsbsO{\typ_n}}$. Given
$\tvarS\in\SeqNR{\Nam}$ and $\typS\in\Seq{\Typ}$ such that
$\seqlen{\tvarS}=\seqlen{\typS}$, we may write
$\tsbs{\typ}{\setST{\tupII{\tvar_i}{\typ_i}}{1\leq i\leq n}}$ as just
$\tsbslash{\typ}{\tvarS}{\typS}$.

%The position of a type occurrence in a type or expression can be identified by
%a sequence of natural numbers that describes a path through the tree. We
%define the set of all possible positions as
%\[
%\Pos=\Seq{\N}
%\]

%The relation
%$\pentrel{\istsbsatY}{(\Typ+\Expr)}{\Typ}{\Typ}{\Pos}{(\Typ+\Expr)}$ captures
%that $\tOe'$ is the result of substituting a type $\typI$ with a type $\typI'$
%at position $\pos$ in a type or expression $\tOe$ (written
%$\istsbsatO{\tOe}{\pos}{\tOe'}$)
%\[
%\begin{lrulelist}
%\lruleO{\istsbsatO{\typI}{\seqO}{\typI'}}
%\lruleI{\istsbsatO{\tinstO}{\cons{i}{\pos}}{\tinst{\tnam}{\seqTF{\typ_i'}}}}
%       {\istsbsatO{\typ_i}{\pos}{\typ_i'}}
%\lruleI{\istsbsatO{(\tarrO)}{\cons{1}{\pos}}{(\tarr{\typ_1'}{\typ_2})}}
%       {\istsbsatO{\typ_1}{\pos}{\typ_1'}}
%\lruleI{\istsbsatO{(\tarrO)}{\cons{2}{\pos}}{(\tarr{\typ_1}{\typ_2'})}}
%       {\istsbsatO{\typ_2}{\pos}{\typ_2'}}
%\lruleI{\istsbsatO{(\trecO)}{\cons{i}{\pos}}{(\trecTF{\fnam_i}{\typ_i'})}}
%       {\istsbsatO{\typ_i}{\pos}{\typ_i'}}
%\lruleI{\istsbsatO{(\tsumO)}{\cons{i}{\pos}}{(\tsumTF{\cnam_i}{\typ_i'})}}
%       {\istsbsatO{\typ_i}{\pos}{\typ_i'}}
%\lruleI{\istsbsatO{(\tsubO)}{\cons{0}{\pos}}{(\tsub{\typ'}{\tspred})}}
%       {\istsbsatO{\typ}{\pos}{\typ'}}
%\lruleI{\istsbsatO{(\tsubO)}{\cons{1}{\pos}}{(\tsub{\typ}{\tspred'})}}
%       {\istsbsatO{\tspred}{\pos}{\tspred'}}
%\lruleI{\istsbsatO{(\tquotO)}{\cons{0}{\pos}}{(\tquot{\typ'}{\tqpred})}}
%       {\istsbsatO{\typ}{\pos}{\typ'}}
%\lruleI{\istsbsatO{(\tquotO)}{\cons{1}{\pos}}{(\tquot{\typ}{\tqpred'})}}
%       {\istsbsatO{\tqpred}{\pos}{\tqpred'}}
%\end{lrulelist}
%\]
%\[
%\begin{lrulelist}
%\lruleI{\istsbsatO{\opO}{\cons{i}{\pos}}{\op{\onam}{\seqTF{\typ_i'}}}}
%       {\istsbsatO{\typ_i}{\pos}{\typ_i'}}
%\lruleI{\istsbsatO{(\appO)}{\cons{1}{\pos}}{(\app{\expr_1'}{\expr_2})}}
%       {\istsbsatO{\expr_1}{\pos}{\expr_1'}}
%\lruleI{\istsbsatO{(\appO)}{\cons{2}{\pos}}{(\app{\expr_1}{\expr_2'})}}
%       {\istsbsatO{\expr_2}{\pos}{\expr_2'}}
%\lruleI{\istsbsatO{(\absO)}{\cons{0}{\pos}}{(\abs{\var}{\typ'}{\expr})}}
%       {\istsbsatO{\typ}{\pos}{\typ'}}
%\lruleI{\istsbsatO{(\absO)}{\cons{1}{\pos}}{(\abs{\var}{\typ}{\expr'})}}
%       {\istsbsatO{\expr}{\pos}{\expr'}}
%\lruleI{\istsbsatO{(\eqO)}{\cons{1}{\pos}}{(\eq{\expr_1'}{\expr_2})}}
%       {\istsbsatO{\expr_1}{\pos}{\expr_1'}}
%\lruleI{\istsbsatO{(\eqO)}{\cons{2}{\pos}}{(\eq{\expr_1}{\expr_2'})}}
%       {\istsbsatO{\expr_2}{\pos}{\expr_2'}}
%\lruleI{\istsbsatO{(\iifO)}{\cons{0}{\pos}}{(\iif{\expr_0'}{\expr_1}{\expr_2})}}
%       {\istsbsatO{\expr_0}{\pos}{\expr_0'}}
%\lruleI{\istsbsatO{(\iifO)}{\cons{1}{\pos}}{(\iif{\expr_0}{\expr_1'}{\expr_2})}}
%       {\istsbsatO{\expr_1}{\pos}{\expr_1'}}
%\lruleI{\istsbsatO{(\iifO)}{\cons{2}{\pos}}{(\iif{\expr_0}{\expr_1}{\expr_2'})}}
%       {\istsbsatO{\expr_2}{\pos}{\expr_2'}}
%\lruleI{\istsbsatO{(\descO)}{\cons{0}{\pos}}{(\desc{\var}{\typ'}{\expr})}}
%       {\istsbsatO{\typ}{\pos}{\typ'}}
%\lruleI{\istsbsatO{(\descO)}{\cons{1}{\pos}}{(\desc{\var}{\typ}{\expr'})}}
%       {\istsbsatO{\expr}{\pos}{\expr'}}
%\lruleI{\istsbsatO{\projO}{\cons{0}{\pos}}{\proj{\expr'}{\fnam}}}
%       {\istsbsatO{\expr}{\pos}{\expr'}}
%\lruleI{\istsbsatO{(\embedO)}
%                  {\cons{i}{\pos}}
%                  {(\embed{\tsumTF{\cnam_i}{\typ_i'}}{\cnam_j})}}
%       {\istsbsatO{\typ_i}{\pos}{\typ_i'}}
%\lruleI{\istsbsatO{\quotO}{\cons{0}{\pos}}{\quot{\tqpred'}}}
%       {\istsbsatO{\tqpred}{\pos}{\tqpred'}}
%\end{lrulelist}
%\]

\subsubsection{Expression substitutions}

The function $\func{\esbsY}{\Expr\times\Nam\times\Expr}{\Expr}$ substitutes a
variable $\varI$ with an expression $\exprI$ in an expression $\expr$ (written
$\esbsO{\expr}$)
\[
\begin{eqlistC}
\eqitem{\esbsO{\var}}
       {\cond{\varI=\var}{\exprI}{\var}}
\eqitem{\esbsO{\opO}}
       {\opO}
\eqitem{\esbsO{(\appO)}}
       {\app{\esbsO{\expr_1}}{\esbsO{\expr_2}}}
\eqitem{\esbsO{(\absO)}}
       {\cond{\varI=\var}{\absO}{\abs{\var}{\typ}{\esbsO{\expr}}}}
\eqitem{\esbsO{(\eqO)}}
       {\eq{\esbsO{\expr_1}}{\esbsO{\expr_2}}}
\eqitem{\esbsO{(\iifO)}}
       {\iif{\esbsO{\expr_0}}{\esbsO{\expr_1}}{\esbsO{\expr_2}}}
\eqitem{\esbsO{(\descopO)}}
       {\descopO}
\eqitem{\esbsO{(\pjop{\fnam})}}
       {\pjop{\fnam}}
\eqitem{\esbsO{(\emb{\cnam})}}
       {\emb{\cnam}}
\eqitem{\esbsO{\quo}}
       {\quo}
\end{eqlistC}
\]
No substitution is performed in the expressions contained in types contained
in expressions because, as already mentioned, such inner expressions have no
free variables in well-formed types.

%Given variables $\varS\in\SeqNR{\Nam}$ and expressions $\exprIS\in\Seq{\Expr}$
%with $\seqlen{\varS}=\seqlen{\exprIS}$, we define the substitution of $\varS$
%with $\exprIS$ in expressions as
%\[
%\begin{eqlistC}
%\eqitem{\esbsO{\opO}}
%       {\opO}
%\eqitem{\esbsO{\var}}
%       {\cond{\var=\var_i}{\exprI_i}{\var}}
%\eqitem{\esbsO{(\appO)}}
%       {\app{\esbsO{\expr_1}}{\esbsO{\expr_2}}}
%\eqitem{\esbsO{(\absO)}}
%       {\abs{\var}{\typ}{\esbsO{\expr}}}
%\eqitem{\esbsO{(\eqO)}}
%       {\eq{\esbsO{\expr_1}}{\esbsO{\expr_2}}}
%\eqitem{\esbsO{(\iifO)}}
%       {\iif{\esbsO{\expr_0}}{\esbsO{\expr_1}}{\esbsO{\expr_2}}}
%\eqitem{\esbsO{\recO}}
%       {\rec{i}{\fnam_i}{\esbsO{\expr_i}}}
%\eqitem{\esbsO{(\projO)}}
%       {\proj{\esbsO{\expr}}{\fnam}}
%\eqitem{\esbsO{\left(\embedO\right)}}
%       {\embedO}
%\eqitem{\esbsO{\relxO}}
%       {\relx{\esbsO{\tspred}}}
%\eqitem{\esbsO{(\restrO)}}
%       {\restr{\esbsO{\tspred}}{\esbsO{\expr}}}
%\eqitem{\esbsO{\quotO}}
%       {\quot{\esbsO{\tqpred}}}
%\eqitem{\esbsO{(\choosO)}}
%       {\choos{\esbsO{\tqpred}}{\esbsO{\expr}}}
%\eqitem{\esbsO{(\caseO)}}
%       {\case{\esbsO{\expr}}{i}{\pat_i}{\esbsO{\expr_i}}}
%\eqitem{\esbsO{(\letrecO)}}
%       {\letrec{\var}{\typ}{\esbsO{\expr}}{\esbsO{\expr'}}}
%\end{eqlistC}
%\]

The function $\func{\cvarvY}{\Expr\times\Nam}{\Setf{\Nam}}$ returns the
variables that would be captured if a variable $\varI$ were substituted with
those variables in an expression $\expr$ (i.e.\ all the variables bound in
$\expr$ at the free occurrences of $\varI$ in $\expr$)
\[
\begin{eqlist}
\eqitem{\cvarvO{\var}}
       {\emptyset}
\eqitem{\cvarvO{\opO}}
       {\emptyset}
\eqitem{\cvarvO{\appO}}
       {\cvarvO{\expr_1}\cup\cvarvO{\expr_2}}
\eqitem{\cvarvO{\absO}}
       {\cond{\varI\in\efvar{\expr}-\setI{\var}}
             {\setI{\var}\cup\cvarvO{\expr}}
             {\emptyset}}
\eqitem{\cvarvO{\eqO}}
       {\cvarvO{\expr_1}\cup\cvarvO{\expr_2}}
\eqitem{\cvarvO{\iifO}}
       {\cvarvO{\expr_0}\cup\cvarvO{\expr_1}\cup\cvarvO{\expr_2}}
\eqitem{\cvarvO{\descopO}}
       {\emptyset}
\eqitem{\cvarvO{\pjop{\fnam}}}
       {\emptyset}
\eqitem{\cvarvO{\emb{\cnam}}}
       {\emptyset}
\eqitem{\cvarvO{\quo}}
       {\emptyset}
\end{eqlist}
\]

The relation $\ternrel{\esbsokY}{\Expr}{\Nam}{\Expr}$ captures the condition
that the substitution $\esbsO{\expr}$ causes no free variables in $\exprI$ to
be captured
\[
\esbsok{\expr}{\varI}{\exprI}\IFF\efvar{\exprI}\cap\cvarvO{\expr}=\emptyset
\]

%Similarly to type occurrences in types and expressions, the position of an
%expression occurrence in an expression can be identified by a sequence of
%natural numbers that describes a path through the tree, i.e.\ an element of
%$\Pos$ (defined earlier).

%The relation $\pentrel{\isesbsatY}{\Expr}{\Expr}{\Expr}{\Pos}{\Expr}$ captures
%that $\expr'$ is the result of substituting an expression $\exprI$ with an
%expression $\exprI'$ at position $\pos$ in an expression $\expr$ (written
%$\isesbsatO{\expr}{\pos}{\expr'}$)
%\[
%\begin{lrulelist}
%\lruleO{\isesbsatO{\exprI}{\seqO}{\exprI'}}
%\lruleI{\isesbsatO{(\appO)}{\cons{1}{\pos}}{(\app{\expr_1'}{\expr_2})}}
%       {\isesbsatO{\expr_1}{\pos}{\expr_1'}}
%\lruleI{\isesbsatO{(\appO)}{\cons{2}{\pos}}{(\app{\expr_1}{\expr_2'})}}
%       {\isesbsatO{\expr_2}{\pos}{\expr_2'}}
%\lruleI{\isesbsatO{(\absO)}{\cons{0}{\pos}}{(\abs{\var}{\typ}{\expr'})}}
%       {\isesbsatO{\expr}{\pos}{\expr'}}
%\lruleI{\isesbsatO{(\eqO)}{\cons{1}{\pos}}{(\eq{\expr_1'}{\expr_2})}}
%       {\isesbsatO{\expr_1}{\pos}{\expr_1'}}
%\lruleI{\isesbsatO{(\eqO)}{\cons{2}{\pos}}{(\eq{\expr_1}{\expr_2'})}}
%       {\isesbsatO{\expr_2}{\pos}{\expr_2'}}
%\lruleI{\isesbsatO{(\iifO)}{\cons{0}{\pos}}{(\iif{\expr_0'}{\expr_1}{\expr_2})}}
%       {\isesbsatO{\expr_0}{\pos}{\expr_0'}}
%\lruleI{\isesbsatO{(\iifO)}{\cons{1}{\pos}}{(\iif{\expr_0}{\expr_1'}{\expr_2})}}
%       {\isesbsatO{\expr_1}{\pos}{\expr_1'}}
%\lruleI{\isesbsatO{(\iifO)}{\cons{2}{\pos}}{(\iif{\expr_0}{\expr_1}{\expr_2'})}}
%       {\isesbsatO{\expr_2}{\pos}{\expr_2'}}
%\lruleI{\isesbsatO{(\descO)}{\cons{0}{\pos}}{(\desc{\var}{\typ}{\expr'})}}
%       {\isesbsatO{\expr}{\pos}{\expr'}}
%\lruleI{\isesbsatO{\projO}{\cons{0}{\pos}}{\proj{\expr'}{\fnam}}}
%       {\isesbsatO{\expr}{\pos}{\expr'}}
%\end{lrulelist}
%\]

%%\[
%%\begin{eqlist}
%%\eqitem{\osbsO{\exprI}{\seqO}}
%%       {\exprI'}
%%\eqitem{\osbsO{(\appO)}{\cons{1}{\pos}}}
%%       {\app{\osbsO{\expr_1}{\pos}}{\expr_2}}
%%\eqitem{\osbsO{(\appO)}{\cons{2}{\pos}}}
%%       {\app{\expr_1}{\osbsO{\expr_2}{\pos}}}
%%\eqitem{\osbsO{(\absO)}{\cons{0}{\pos}}}
%%       {\abs{\var}{\typ}{\osbsO{\expr}{\pos}}}
%%\eqitem{\osbsO{(\eqO)}{\cons{1}{\pos}}}
%%       {\eq{\osbsO{\expr_1}{\pos}}{\expr_2}}
%%\eqitem{\osbsO{(\eqO)}{\cons{2}{\pos}}}
%%       {\eq{\expr_1}{\osbsO{\expr_2}{\pos}}}
%%\eqitem{\osbsO{(\iifO)}{\cons{0}{\pos}}}
%%       {\iif{\osbsO{\expr_0}{\pos}}{\expr_1}{\expr_2}}
%%\eqitem{\osbsO{(\iifO)}{\cons{1}{\pos}}}
%%       {\iif{\expr_0}{\osbsO{\expr_1}{\pos}}{\expr_2}}
%%\eqitem{\osbsO{(\iifO)}{\cons{2}{\pos}}}
%%       {\iif{\expr_0}{\expr_1}{\osbsO{\expr_2}{\pos}}}
%%\eqitem{\osbsO{\recO}{\cons{i}{\pos}}}
%%       {\recFIT{\fnam_1}{\expr_1}
%%               {\fnam_i}{\osbsO{\expr_i}{\pos}}
%%               {\fnam_n}{\expr_n}}
%%\eqitem{\osbsO{(\projO)}{\cons{0}{\pos}}}
%%       {\proj{\osbsO{\expr}{\pos}}{\fnam}}
%%\eqitem{\osbsO{(\restrO)}{\cons{0}{\pos}}}
%%       {\restr{\tspred}{\osbsO{\expr}{\pos}}}
%%\eqitem{\osbsO{(\choosO)}{\cons{0}{\pos}}}
%%       {\choos{\tqpred}{\osbsO{\expr}{\pos}}}
%%\eqitem{\osbsO{(\caseO)}{\cons{0}{\pos}}}
%%       {\case{\osbsO{\expr}{\pos}}{i}{\pat_i}{\expr_i}}
%%\eqitem{\osbsO{(\caseO)}{\cons{i}{\pos}}}
%%       {\caseFIT{\expr}{\pat_1}{\expr_1}
%%                       {\pat_i}{\osbsO{\expr_i}{\pos}}
%%                       {\pat_n}{\expr_n}}
%%\eqitem{\osbsO{(\letrecO)}{\cons{0}{\pos}}}
%%       {\letrec{\var}{\typ}{\osbsO{\expr}{\pos}}{\expr'}}
%%\eqitem{\osbsO{(\letrecO)}{\cons{1}{\pos}}}
%%       {\letrec{\var}{\typ}{\expr}{\osbsO{\expr'}{\pos}}}
%%\end{eqlist}
%%\]

%The relation $\binrel{\posineY}{\Pos}{\Expr}$ captures legal positions in
%expressions
%\[
%\posineO\IFF\EXISTS{\expr',\exprI,\exprI'}{\isesbsatO{\expr}{\pos}{\expr'}}
%\]

%%The function
%%$\func{\expratY}
%%      {\setST{\tupII{\expr}{\pos}\in\Expr\times\Pos}{\posine{\pos}{\expr}}}
%%      {\Expr}$
%%returns the subexpression at a given position in a superexpression
%%\[
%%\isesbsatO{\expr}{\pos}{\expr'}\IMPLIES\exprat{\expr}{\pos}=\exprI
%%\]

%%The function
%%$\func{\esbsatY}
%%      {\setST{\tupIV{\expr}{\exprI}{\exprI'}{\pos}
%%              \in\Expr\times\Expr\times\Expr\times\Pos}
%%             {\EXISTS{\expr'}{\isesbsatO{\expr}{\pos}{\expr'}}}}
%%      {\Expr}$
%%substitutes an expression $\exprI$ with an expression $\exprI'$ at position
%%$\pos$ in an expression $\expr$ (written $\esbsatO{\expr}{\pos}$)
%%\[
%%\isesbsatO{\expr}{\pos}{\expr'}\IMPLIES\esbsatO{\expr}{\pos}=\expr'
%%\]

%The function
%$\func{\cvarpY}
%      {\setST{\tupII{\expr}{\pos}\in\Expr\times\Pos}{\posineO}}{\Setf{\Nam}}$
%returns the variables that would be captured if they were substituted in an
%expression $\expr$ at position $\pos$ (i.e.\ all the variables bound in $\expr$
%at position $\pos$)
%\[
%\begin{eqlist}
%\eqitem{\cvarp{\expr}{\seqO}}
%       {\emptyset}
%\eqitem{\cvarp{\appO}{(\cons{1}{\pos})}}
%       {\cvarp{\expr_1}{\pos}}
%\eqitem{\cvarp{\appO}{(\cons{2}{\pos})}}
%       {\cvarp{\expr_2}{\pos}}
%\eqitem{\cvarp{\absO}{(\cons{0}{\pos})}}
%       {\setI{\var}\cup\cvarp{\expr}{\pos}}
%\eqitem{\cvarp{\eqO}{(\cons{1}{\pos})}}
%       {\cvarp{\expr_1}{\pos}}
%\eqitem{\cvarp{\eqO}{(\cons{2}{\pos})}}
%       {\cvarp{\expr_2}{\pos}}
%\eqitem{\cvarp{\iifO}{(\cons{0}{\pos})}}
%       {\cvarp{\expr_0}{\pos}}
%\eqitem{\cvarp{\iifO}{(\cons{1}{\pos})}}
%       {\cvarp{\expr_1}{\pos}}
%\eqitem{\cvarp{\iifO}{(\cons{2}{\pos})}}
%       {\cvarp{\expr_2}{\pos}}
%\eqitem{\cvarp{\descO}{(\cons{0}{\pos})}}
%       {\setI{\var}\cup\cvarp{\expr}{\pos}}
%\eqitem{\cvarp{\projO}{(\cons{0}{\pos})}}
%       {\cvarp{\expr}{\pos}}
%\end{eqlist}
%\]

%The relation $\quatrel{\esbsatokY}{\Expr}{\Expr}{\Expr}{\Pos}$ captures the
%condition that the substitution $\isesbsatO{\expr}{\pos}{\expr'}$ is defined
%(for some $\expr'$), affects no free variables in $\exprI$ that are bound in
%$\expr$, and causes no free variables in $\exprI'$ to be captured
%\[
%\esbsatok{\expr}{\exprI}{\exprI'}{\pos}
%\IFF
%\EXISTS{\expr'}
%       {\isesbsatO{\expr}{\pos}{\expr'}
%        \AND
%        \efvar{\exprI}\cap\cvarp{\expr}{\pos}=\emptyset
%        \AND
%        \efvar{\exprI'}\cap\cvarp{\expr}{\pos}=\emptyset}
%\]
