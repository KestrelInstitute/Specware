\section{Properties}
\label{props}

\subsection{Additional notions}

In order to prove certain properties of the syntax and proof theory formalized
in \secref{syntax} and \secref{proofth}, we introduce some additional notions.
Those notions have not been introduced earlier because they are not needed to
define the \MS\ logic, but only to express and prove properties about it.

%We extend $\efvarY$ to context elements and contexts
%\[
%\begin{eqlist}
%\eqitem{\efvar{\tdecO}} {\emptyset}
%\eqitem{\efvar{\odecO}} {\emptyset}
%\eqitem{\efvar{\tdefO}} {\emptyset}
%%\eqitem{\efvar{\odefO}} {\efvar{\expr}}
%\eqitem{\efvar{\axO}}   {\efvar{\expr}}
%\eqitem{\efvar{\lemO}}  {\efvar{\expr}}
%\eqitem{\efvar{\tvdecO}}{\emptyset}
%\eqitem{\efvar{\vdecO}} {\emptyset}
%\end{eqlist}
%\]
%\[
%\begin{eqlist}
%\eqitem{\efvar{\mtcx}}      {\emptyset}
%\eqitem{\efvar{\cx_1,\cx_2}}{\efvar{\cx_1}\cup\efvar{\cx_2}}
%\end{eqlist}
%\]

We extend expression substitution to context elements and contexts
\[
\begin{eqlistC}
\eqitem{\esbsO{(\tdecO)}} {\tdecO}
\eqitem{\esbsO{(\odecO)}} {\odecO}
\eqitem{\esbsO{(\tdefO)}} {\tdefO}
%\eqitem{\esbsO{(\odefO)}} {\odef{\tvarS}{\onam}{\esbsO{\expr}}}
\eqitem{\esbsO{(\axO)}}   {\ax{\tvarS}{\esbsO{\expr}}}
\eqitem{\esbsO{(\lemO)}}  {\lem{\tvarS}{\esbsO{\expr}}}
\eqitem{\esbsO{(\tvdecO)}}{\tvdecO}
\eqitem{\esbsO{(\vdecO)}} {\vdecO}
\end{eqlistC}
\]
\[
\begin{eqlistC}
\eqitem{\esbsO{\mtcx}}        {\mtcx}
\eqitem{\esbsO{(\cx_1,\cx_2)}}{\esbsO{\cx_1},\esbsO{\cx_2}}
\end{eqlistC}
\]
Note that $\esbsO{(\vdecO)}=\vdecO$ even if $\varI=\var$, because a variable
declaration ``works'' more or less like a binder (and the $\var$ in $\absO$ is
unchanged under substitution).

The function $\func{\ftvarY}{\Typ+\Expr+\Cxel+\Cx}{\Setf{\Nam}}$ returns the
free type variables of a type, expression, or context (element)
\[
\begin{eqlist}
\eqitem{\ftvar{\bool}}  {\emptyset}
\eqitem{\ftvar{\tvar}}  {\setI{\tvar}}
\eqitem{\ftvar{\tinstO}}{\bigcup_i\ftvar{\typ_i}}
\eqitem{\ftvar{\tarrO}} {\ftvar{\typ_1}\cup\ftvar{\typ_2}}
\eqitem{\ftvar{\trecO}} {\bigcup_i\ftvar{\typ_i}}
%\eqitem{\ftvar{\tsumO}} {\bigcup_i\ftvar{\typ_i}}
\eqitem{\ftvar{\tsubO}} {\ftvar{\typ}\cup\ftvar{\tspred}}
%\eqitem{\ftvar{\tquotO}}{\ftvar{\typ}\cup\ftvar{\tqpred}}
\end{eqlist}
\]
\[
\begin{eqlist}
\eqitem{\ftvar{\var}}    {\emptyset}
\eqitem{\ftvar{\opO}}    {\bigcup_i\ftvar{\typ_i}}
\eqitem{\ftvar{\appO}}   {\ftvar{\expr_1}\cup\ftvar{\expr_2}}
\eqitem{\ftvar{\absO}}   {\ftvar{\typ}\cup\ftvar{\expr}}
\eqitem{\ftvar{\eqO}}    {\ftvar{\expr_1}\cup\ftvar{\expr_2}}
\eqitem{\ftvar{\iifO}}   {\ftvar{\expr_0}\cup
                          \ftvar{\expr_1}\cup\ftvar{\expr_2}}
\eqitem{\ftvar{\descopO}}{\ftvar{\typ}}
\eqitem{\ftvar{\projopO}}{\ftvar{\trecO}}
%\eqitem{\ftvar{\embedO}} {\ftvar{\tsumO}}
%\eqitem{\ftvar{\quotO}}  {\ftvar{\tquotO}}
\end{eqlist}
\]
\[
\begin{eqlist}
\eqitem{\ftvar{\tdecO}} {\emptyset}
\eqitem{\ftvar{\odecO}} {\ftvar{\typ}-\tvarS}
\eqitem{\ftvar{\tdefO}} {\ftvar{\typ}-\tvarS}
\eqitem{\ftvar{\axO}}   {\ftvar{\expr}-\tvarS}
\eqitem{\ftvar{\lemO}}  {\ftvar{\expr}-\tvarS}
\eqitem{\ftvar{\tvdecO}}{\emptyset}
\eqitem{\ftvar{\vdecO}} {\ftvar{\typ}}
\end{eqlist}
\]
\[
\begin{eqlist}
\eqitem{\ftvar{\mtcx}}      {\emptyset}
\eqitem{\ftvar{\cx_1,\cx_2}}{\ftvar{\cx_1}\cup\ftvar{\cx_2}}
\end{eqlist}
\]
Note that the type variables that occur in a type or expression are always
free; the only bound type variables are the $\tvarS$ in context elements.

We extend type substitution to context elements, contexts, and type
substitution themselves
\[
\begin{eqlistC}
\eqitem{\tsbsO{(\tdecO)}}
       {\tdecO}
\eqitem{\tsbsO{(\odecO)}}
       {\odec{\onam}{\tvarS}{\tsbs{\typ}{\funrestrc{\tsbsmap}{\tvarS}}}}
\eqitem{\tsbsO{(\tdefO)}}
       {\tdef{\tnam}{\tvarS}{\tsbs{\typ}{\funrestrc{\tsbsmap}{\tvarS}}}}
\eqitem{\tsbsO{(\axO)}}
       {\ax{\tvarS}{\tsbs{\expr}{\funrestrc{\tsbsmap}{\tvarS}}}}
\eqitem{\tsbsO{(\lemO)}}
       {\lem{\tvarS}{\tsbs{\expr}{\funrestrc{\tsbsmap}{\tvarS}}}}
\eqitem{\tsbsO{(\tvdecO)}}
       {\tvdecO}
\eqitem{\tsbsO{(\vdecO)}}
       {\vdec{\var}{\tsbsO{\typ}}}
\end{eqlistC}
\]
\[
\begin{eqlistC}
\eqitem{\tsbsO{\mtcx}}        {\mtcx}
\eqitem{\tsbsO{(\cx_1,\cx_2)}}{\tsbsO{\cx_1},\tsbsO{\cx_2}}
\end{eqlistC}
\]
\[
\tsbs{\tsbsmap}{\tsbsmap'} =
\setST{\tupII{\tvar}{\tsbs{\tsbsmap(\tvar)}{\tsbsmap'}}}{\tvar\in\dom{\tsbsmap}}
\]

The function $\func{\ctvarY}{(\Cxel+\Cx)\times\Nam}{\Setf{\Nam}}$ returns the
type variables that would be captured if a type variable $\tvarI$ were
substituted with those type variables in a context (element), i.e.\ all the
variables bound in the (context) element at the free occurrences of $\tvarI$
in the context (element)
\[
\begin{eqlist}
\eqitem{\ctvarO{\tdecO}}
       {\emptyset}
\eqitem{\ctvarO{\odecO}}
       {\cond{\tvarI\in\ftvar{\typ}-\tvarS}{\tvarS}{\emptyset}}
\eqitem{\ctvarO{\tdefO}}
       {\cond{\tvarI\in\ftvar{\typ}-\tvarS}{\tvarS}{\emptyset}}
\eqitem{\ctvarO{\axO}}
       {\cond{\tvarI\in\ftvar{\expr}-\tvarS}{\tvarS}{\emptyset}}
\eqitem{\ctvarO{\lemO}}
       {\cond{\tvarI\in\ftvar{\expr}-\tvarS}{\tvarS}{\emptyset}}
\eqitem{\ctvarO{\tvdecO}}
       {\emptyset}
\eqitem{\ctvarO{\vdecO}}
       {\emptyset}
\end{eqlist}
\]
\[
\begin{eqlist}
\eqitem{\ctvarO{\mtcx}}        {\emptyset}
\eqitem{\ctvarO{(\cx_1,\cx_2)}}{\ctvarO{\cx_1}\cup\ctvarO{\cx_2}}
\end{eqlist}
\]

The relation $\binrel{\tsbsokY}{\Cx}{(\Funcp{\Nam}{\Typ})}$ captures the
condition that the substitution $\tsbsO{\cx}$ causes no free type variables in
$\cx$ to be captured and does not substitute any type variable declared in
$\cx$
\[
\tsbsok{\cx}{\tsbsmap}
 \IFF
(\FORALL{\tvar\in\dom{\tsbsmap}}
        {\ftvar{\tsbsmap(\tvar)}\cap\ctvar{\cx}{\tvar}=\emptyset})
 \AND
\dom{\tsbsmap}\cap\cxtvarO=\emptyset
\]
Given $\tvarS\in\SeqNR{\Nam}$ and $\typS\in\Seq{\Typ}$ such that
$\seqlen{\tvarS}=\seqlen{\typS}$, we may write
$\tsbsok{\cx}{\setST{\tupII{\tvar_i}{\typ_i}}{1\leq i\leq n}}$ as just
$\tsbslashok{\cx}{\tvarS}{\typS}$.


\subsection{Syntactic properties}

Expression substitution only takes place at the free occurrences of the
variable that is substituted. Thus, if the variable is not free in the
expression, substitution causes no change and no variable capture:

\begin{theorem}\label{thm-esbs-not-free}
\[
\varI\not\in\efvar{\expr}\IMPLIES
\cvarvO{\expr}=\emptyset\AND
\esbsokO{\expr}\AND
\esbsO{\expr}=\expr
\]
\end{theorem}

Since expression substitution only substitutes the free occurrences of the
variable, the variable itself is not captured at those occurrences:

\begin{theorem}\label{thm-var-not-capt}
\[
\varI\not\in\cvarvO{\expr}
\]
\end{theorem}

It is always possible to substitute a variable with itself and the
substitution causes no change:

\begin{theorem}\label{thm-esbs-id}
\[
\esbsok{\expr}{\varI}{\varI}\AND
\esbs{\expr}{\varI}{\varI}=\expr
\]
\end{theorem}

The following four theorems say that if a substitution causes no capture in a
compound expression, it does not cause capture in the component expressions
either:

\begin{theorem}\label{thm-esbsok-app}
\[
\esbsokO{\appO}\IMPLIES\esbsokO{\expr_1}\AND\esbsokO{\expr_2}
\]
\end{theorem}

\begin{theorem}\label{thm-esbsok-eq}
\[
\esbsokO{\eqO}\IMPLIES\esbsokO{\expr_1}\AND\esbsokO{\expr_2}
\]
\end{theorem}

\begin{theorem}\label{thm-esbsok-if}
\[
\esbsokO{\iifO}\IMPLIES
\esbsokO{\expr_0}\AND\esbsokO{\expr_1}\AND\esbsokO{\expr_2}
\]
\end{theorem}

\begin{theorem}\label{thm-esbsok-abs}
\[
\esbsokO{\absO}\AND\varI\neq\var\IMPLIES\esbsokO{\expr}
\]
\end{theorem}

In the last theorem, the condition $\varI\neq\var$ is used in the proof and is
in fact necessary, because in general
$\esbsokO{\absO}\NIMPLIES\esbsokO{\expr}$, as shown by the counter-example
$\varI=\var$, $\expr=\abs{\varII}{\typ}{\var}$, and $\exprI=\varII$.

If we substitute $\varI$ with $\exprI$ in $\expr$, we remove $\varI$ from the
free variables of $\expr$ and add the free variables of $\exprI$. This is
actually true only if $\varI$ is free in $\expr$ (otherwise the substitution
causes no change) and no variable is captured (otherwise the captured
variables of $\exprI$ would not contribute to the free variables of the result
of substitution):

\begin{theorem}\label{thm-efvar-of-esbs}
\[
\varI\in\efvar{\expr}\AND
\esbsokO{\expr}\IMPLIES
\efvar{\esbsO{\expr}}=(\efvar{\expr}-\setI{\varI})\cup\efvar{\exprI}
\]
\end{theorem}

If we substitute $\varI$ with an expression $\exprI$ that does not have
$\varI$ among its free variables, the resulting expression does not have
$\varI$ among its free variables (provided that the substitution captured no
variables):

\begin{theorem}\label{thm-var-not-in-esbs-of-var}
\[
\esbsokO{\expr}\AND\varI\not\in\efvar{\exprI}\IMPLIES
\varI\not\in\efvar{\esbsO{\expr}}
\]
\end{theorem}

Two expression substitutions commute if their variables do not ``interact'':

\begin{theorem}\label{thm-esbs-comm}
\[
\varI\neq\varI'\AND
\varI\not\in\efvar{\exprI'}\AND
\varI'\not\in\efvar{\exprI}\IMPLIES
\esbs{\esbs{\expr}{\varI}{\exprI}}{\varI'}{\exprI'}=
\esbs{\esbs{\expr}{\varI'}{\exprI'}}{\varI}{\exprI}
\]
\end{theorem}

If we substitute first $\varI$ with $\exprI$ and then $\varI$ with $\exprI'$,
we obtain the same result as directly substituting $\varI$ with
$\esbs{\exprI}{\varI}{\exprI'}$:

\begin{theorem}\label{thm-esbs-esbs-of-same-var}
\[
\esbs{\esbs{\expr}{\varI}{\exprI}}{\varI}{\exprI'}=
\esbs{\expr}{\varI}{\esbs{\exprI}{\varI}{\exprI'}}
\]
\end{theorem}

Variable renaming is idempotent:

\begin{theorem}\label{thm-var-rename-idemp}
\[
\esbsren{\esbsren{\expr}}=\esbsren{\expr}
\]
\end{theorem}

Substituting $\varII$ with $\exprI$ (such that no capture takes place) and
then renaming $\varI$ to $\varI'$ is equivalent to substituting $\varII$ with
the renaming applied to $\exprI$ (which renames the free occurrences of
$\varI$ in $\exprI$) and then applying the renaming to the result (which
renames the free occurrences of $\varI$ in the original expression $\expr$):

\begin{theorem}\label{thm-esbs-then-rename}
\[
\esbsok{\expr}{\varII}{\exprI}\IMPLIES
\esbsren{\esbs{\expr}{\varII}{\exprI}}=
\esbsren{\esbs{\expr}{\varII}{\esbsren{\exprI}}}
\]
\end{theorem}

In an expression $\expr$, renaming $\varI$ to $\varI'$ and then back $\varI'$
to $\varI$ leaves $\expr$ unchanged, provided that $\varI'$ is not captured
(otherwise the captured occurrences would not be renamed back to $\varI$) and
does not already occur free in $\expr$ (otherwise we do not obtain $\expr$ at
the end, because all free occurrences of $\varI'$ disappear when we rename
$\varI'$ to $\varI$):

\begin{theorem}\label{thm-esbs-inverse}
\[
\esbsok{\expr}{\varI}{\varI'}\AND
\varI'\not\in\efvar{\expr}\IMPLIES
\esbs{\esbsren{\expr}}{\varI'}{\varI}=\expr
\]
\end{theorem}

If we rename a variable $\varI$ to $\varI'$ in an expression then the
variables captured at the free occurrences of $\varI$ in the original
expression coincide with those captured at the free occurrences of $\varI'$ in
the transformed expression (provided that $\varI'$ is not captured in the
renaming and that $\varI'$ was not already free in the original expression):

\begin{theorem}\label{thm-capt-rename1}
\[
\esbsok{\expr}{\varI}{\varI'}\AND
\varI'\not\in\efvar{\expr}\IMPLIES
\cvarvO{\expr}=\cvarv{\esbsren{\expr}}{\varI'}
\]
\end{theorem}

The following theorem says how captured variables change under expression
substitution, assuming that the substituted variable is free in the expression
(otherwise no change takes place) and that no variable is captured. The
relationship should be quite intuitive when thinking of expressions as trees
and captured variables as the lambda-bound variables encountered along the
paths in the trees. The substitution replaces each free occurrence of $\varI$
in $\expr$ with $\exprI$. Thus, if $\varII$ is free in $\exprI$ we have to add
the variables in the paths to free occurrences of $\varII$ in $\exprI$ to
those in the paths to free occurrences of $\varI$ in $\expr$. If $\varII$ is
not free in $\exprI$, there are no such paths. In addition, if
$\varII\neq\varI$, we add the variables in the paths to the free occurrences
of $\varII$ in $\expr$; if instead $\varII=\varI$, only the free occurrences
of $\varII$ in $\exprI$ count.

\begin{theorem}\label{thm-cvar-of-esbs}
{\rm
\[
\begin{array}{l}
\varI\in\efvar{\expr}
\ \AND\
\esbsokO{\expr}
\ \IMPLIES\\
\cvarv{\esbsO{\expr}}{\varII}=
\cond{\varII\neq\varI}
     {\cvarv{\expr}{\varII}}
     {\emptyset}
\cup
\cond{\varII\in\efvar{\exprI}}
     {\cvarv{\expr}{\varI}\cup\cvarv{\exprI}{\varII}}
     {\emptyset}
\end{array}
\]
}
\end{theorem}

If we rename a variable $\varI$ to $\varI'$ in an expression then the
variables captured at the free occurrences of a third variable $\varII$ in the
original expression coincide with those captured at the free occurrences of
$\varII$ in the transformed expression (provided that $\varI'$ is not captured
in the renaming):

\begin{theorem}\label{thm-capt-rename2}
\[
\esbsok{\expr}{\varI}{\varI'}\AND
\varII\neq\varI\AND
\varII\neq\varI'\IMPLIES
\cvarv{\expr}{\varII}=\cvarv{\esbsren{\expr}}{\varII}
\]
\end{theorem}

Renaming bound variables does not change free variables, provided that the
renaming causes no capture (not only the substitution
$\esbs{\expr}{\var}{\var'}$ must cause no capture, but also $\var'$ must not
occur free in $\expr$, otherwise it would be captured by the top-level
$\abs{\var'}{\typ}{\ldots}$):

\begin{theorem}\label{thm-alpha-same-free-vars}
\[
\var'\not\in\efvar{\expr}\cup\cvarv{\expr}{\var}\IMPLIES
\efvar{\absO}=\efvar{\abs{\var'}{\typ}{\esbs{\expr}{\var}{\var'}}}
\]
\end{theorem}

Type substitution does not change free and captured variables:

\begin{theorem}\label{thm-tsbs-same-free-vars}
\[
\efvar{\tsbsO{\expr}}=\efvar{\expr}
\]
\end{theorem}

\begin{theorem}\label{thm-tsbs-same-capt-vars}
\[
\cvarvO{\tsbsO{\expr}}=\cvarvO{\expr}
\]
\end{theorem}

If we substitute $\varI$ with $\exprI$ in an $\expr$, we add the free type
variables of $\exprI$ to those of $\expr$ if $\varI$ is free in $\expr$
(otherwise we add no type variables):

\begin{theorem}\label{thm-ftvar-of-esbs}
{\rm
\[
\ftvar{\esbsO{\expr}}=
\ftvar{\expr}\cup\cond{\varI\in\efvar{\expr}}{\ftvar{\exprI}}{\emptyset}
\]
}
\end{theorem}

Variable renaming does not change free type variables:

\begin{theorem}\label{thm-rename-same-free-tvars}
\[
\ftvar{\esbsren{\expr}}=\ftvar{\expr}
\]
\end{theorem}

Applying a type substitution $\tsbsmap$ to the result of a variable
substitution $\esbsO{\expr}$ is like applying $\tsbsmap$ to $\expr$ and
$\exprI$ and then performing the variable substitution:

\begin{theorem}\label{thm-tsbs-of-esbs}
\[
\tsbsO{\esbsO{\expr}} = \esbs{\tsbsO{\expr}}{\varI}{\tsbsO{\exprI}}
\]
\end{theorem}

Variable renaming commutes with type substitution:

\begin{theorem}\label{thm-esbs-tsbs-comm}
\[
\tsbsO{\esbsren{\expr}}=\esbsren{\tsbsO{\expr}}
\]
\end{theorem}

Expression substitution does not change the type, op, and (type) variable
declarations in a context, does not change the type definitions in a context,
and applies to the axioms and lemmas in a context:

\begin{theorem}\label{thm-esbs-cx-decl}
\[
\begin{eqlistC}
\eqitem {\cxtnam{\esbsO{\cx}}} {\cxtnamO}
\eqitem {\cxonam{\esbsO{\cx}}} {\cxonamO}
\eqitem {\cxtvar{\esbsO{\cx}}} {\cxtvarO}
\eqitem {\cxvar {\esbsO{\cx}}} {\cxvarO}
\iffitem {\tdecO\in\cx}  {\tdecO\in\esbsO{\cx}}
\iffitem {\odecO\in\cx}  {\odecO\in\esbsO{\cx}}
\iffitem {\tdefO\in\cx}  {\tdefO\in\esbsO{\cx}}
%\impitem {\odefO\in\cx}  {\odef{\tvarS}{\onam}{\esbsO{\expr}}\in\esbsO{\cx}}
\impitem {\axO\in\cx}    {\ax{\tvarS}{\esbsO{\expr}}\in\esbsO{\cx}}
\impitem {\lemO\in\cx}   {\lem{\tvarS}{\esbsO{\expr}}\in\esbsO{\cx}}
\iffitem {\tvdecO\in\cx} {\tvdecO\in\esbsO{\cx}}
\iffitem {\vdecO\in\cx}  {\vdecO\in\esbsO{\cx}}
\end{eqlistC}
\]
\end{theorem}

An empty type substitution causes no change:
\begin{theorem}\label{thm-tsbs-id}
\[
\tOeOc\in\Typ+\Expr+\Cxel+\Cx
\ \ \IMPLIES\ \
\tsbs{\tOeOc}{\emptyset}=\tOeOc
\]
\end{theorem}

In a type substitution, only the type variables that are free in the type,
expression, or context (element) matter (i.e.\ we can eliminate the other type
variables from the domain of the substitution):

\begin{theorem}\label{thm-tsbs-ftvar}
\[
\tOeOc\in\Typ+\Expr+\Cxel+\Cx
\ \ \IMPLIES\ \
\tsbsO{\tOeOc}=\tsbs{\tOeOc}{\funrestr{\tsbsmap}{\ftvar{\tOeOc}}}
\]
\end{theorem}

If none of the type variables substituted by a type substitution appears in
the type, expression, or context (element) to which the type substitution is
applied, then the type, expression, or context (element) is unchanged:

\begin{theorem}\label{thm-tsbs-not-free}
\[
\tOeOc\in\Typ+\Expr+\Cxel+\Cx
\ \ \AND\ \
\dom{\tsbsmap}\cap\ftvar{\tOeOc}=\emptyset
\ \ \IMPLIES\ \
\tsbsO{\tOeOc}=\tOeOc
\]
\end{theorem}

If we apply a type substitution $\tsbsmap$ to a type or expression $\tOe$, the
free type variables that are substituted are replaced by the free type
variables of the types that replace those type variables:

\begin{theorem}\label{thm-ftvar-of-tsbs}
\[
\tOe\in\Typ+\Expr
\ \ \IMPLIES\ \
\ftvar{\tsbsO{\tOe}}
\ \ =\ \
(\ftvar{\tOe}-\dom{\tsbsmap})
\ \ \cup
\bigcup_{\tvarI\in\ftvar{\tOe}\cap\dom{\tsbsmap}}\ftvar{\tsbsmap(\tvarI)}
\]
\end{theorem}

Applying two type substitutions $\tsbsmap$ and $\tsbsmap'$ one after the other
to a type or expression is like applying $\tsbs{\tsbsmap}{\tsbsmap'}$,
provided that $\tsbsmap'$ does not substitute type variables that are free in
the type or expression but are not in the domain of $\tsbsmap$:

\begin{theorem}\label{thm-tsbs-tsbs}
\[
\tOe\in\Typ+\Expr
\ \ \AND\ \
(\ftvar{\tOe}-\dom{\tsbsmap})\cap\dom{\tsbsmap'}=\emptyset
\ \ \IMPLIES\ \
\tsbs{\tsbs{\tOe}{\tsbsmap}}{\tsbsmap'} =
\tsbs{\tOe}{\tsbs{\tsbsmap}{\tsbsmap'}}
\]
\end{theorem}

Applying a type substitutions $\tsbsmap$ followed by another type substitution
$\tsbsmap'$ is like applying $\tsbsmap'$ followed by
$\tsbs{\tsbsmap}{\tsbsmap'}$, under certain conditions:

\begin{theorem}\label{thm-tsbs-tsbs2}
\[
\begin{array}{l}
\tOe\in\Typ+\Expr
\\
\dom{\tsbsmap}\cap\dom{\tsbsmap'}=\emptyset
\\
(\FORALL{\tvarI\in\ftvar{\tOe}\cap\dom{\tsbsmap'}}
        {\ftvar{\tsbsmap'(\tvarI)}\cap\dom{\tsbsmap}=\emptyset})
\\
\ \ \IMPLIES\\
\tsbs{\tsbsO{\tOe}}{\tsbsmap'} =
\tsbs{\tsbs{\tOe}{\tsbsmap'}}{\tsbs{\tsbsmap}{\tsbsmap'}}
\end{array}
\]
\end{theorem}

\subsection{Properties of abbreviations}

The following five theorems show that $\efvarY$,
% $\opsinY$,
$\esbsY$, $\cvarvY$, $\tsbsY$, and $\ftvarY$ ``extend'' to the abbreviations
defined in \secref{syntax} as expected.

\begin{theorem}\label{thm-efvar-abbrev}
\[
\begin{eqlist}
\eqitem{\efvar{\true}}
       {\emptyset}
\eqitem{\efvar{\false}}
       {\emptyset}
\eqitem{\efvar{\negaop}}
       {\emptyset}
\eqitem{\efvar{\conjO}}
       {\efvar{\expr_1}\cup\efvar{\expr_2}}
\eqitem{\efvar{\disjO}}
       {\efvar{\expr_1}\cup\efvar{\expr_2}}
\eqitem{\efvar{\implO}}
       {\efvar{\expr_1}\cup\efvar{\expr_2}}
\eqitem{\efvar{\iiffop}}
       {\emptyset}
\eqitem{\efvar{\iiffO}}
       {\efvar{\expr_1}\cup\efvar{\expr_2}}
\eqitem{\efvar{\neeqO}}
       {\efvar{\expr_1}\cup\efvar{\expr_2}}
\eqitem{\efvar{\descO}}
       {\efvar{\expr}-\setI{\var}}
\eqitem{\efvar{\faopO}}
       {\emptyset}
\eqitem{\efvar{\faO}}
       {\efvar{\expr}-\setI{\var}}
\eqitem{\efvar{\faS{\seqFT{\bnd{\var_1}{\typ_1}}{\bnd{\var_n}{\typ_n}}}{\expr}}}
       {\efvar{\expr}-\varS}
\eqitem{\efvar{\fa{\varS}{\typS}{\expr}}}
       {\efvar{\expr}-\varS}
\eqitem{\efvar{\exopO}}
       {\emptyset}
\eqitem{\efvar{\exO}}
       {\efvar{\expr}-\setI{\var}}
\eqitem{\efvar{\exS{\seqFT{\bnd{\var_1}{\typ_1}}{\bnd{\var_n}{\typ_n}}}{\expr}}}
       {\efvar{\expr}-\varS}
\eqitem{\efvar{\ex{\varS}{\typS}{\expr}}}
       {\efvar{\expr}-\varS}
\eqitem{\efvar{\exIopO}}
       {\emptyset}
\eqitem{\efvar{\exIO}}
       {\efvar{\expr}-\setI{\var}}
\eqitem{\efvar{\projO}}
       {\efvar{\expr}}
\end{eqlist}
\]
\end{theorem}

%\begin{theorem}\label{thm-opsin-abbrev}
%\[
%\begin{eqlist}
%\eqitem{\opsin{\true}}
%       {\emptyset}
%\eqitem{\opsin{\false}}
%       {\emptyset}
%\eqitem{\opsin{\negaop}}
%       {\emptyset}
%\eqitem{\opsin{\conjO}}
%       {\opsin{\expr_1}\cup\opsin{\expr_2}}
%\eqitem{\opsin{\disjO}}
%       {\opsin{\expr_1}\cup\opsin{\expr_2}}
%\eqitem{\opsin{\implO}}
%       {\opsin{\expr_1}\cup\opsin{\expr_2}}
%\eqitem{\opsin{\iiffop}}
%       {\emptyset}
%\eqitem{\opsin{\iiffO}}
%       {\opsin{\expr_1}\cup\opsin{\expr_2}}
%\eqitem{\opsin{\neeqO}}
%       {\opsin{\expr_1}\cup\opsin{\expr_2}}
%\eqitem{\opsin{\descO}}
%       {\opsin{\typ}\cup\opsin{\expr}}
%\eqitem{\opsin{\faopO}}
%       {\opsin{\typ}}
%\eqitem{\opsin{\faO}}
%       {\opsin{\typ}\cup\opsin{\expr}}
%\eqitem{\opsin{\faS{\seqFT{\bnd{\var_1}{\typ_1}}{\bnd{\var_n}{\typ_n}}}{\expr}}}
%       {\opsin{\expr}\cup\bigcup_i\opsin{\typ_i}}
%\eqitem{\opsin{\fa{\varS}{\typS}{\expr}}}
%       {\opsin{\expr}\cup\bigcup_i\opsin{\typ_i}}
%\eqitem{\opsin{\exopO}}
%       {\opsin{\typ}}
%\eqitem{\opsin{\exO}}
%       {\opsin{\typ}\cup\opsin{\expr}}
%\eqitem{\opsin{\exS{\seqFT{\bnd{\var_1}{\typ_1}}{\bnd{\var_n}{\typ_n}}}{\expr}}}
%       {\opsin{\expr}\cup\bigcup_i\opsin{\typ_i}}
%\eqitem{\opsin{\ex{\varS}{\typS}{\expr}}}
%       {\opsin{\expr}\cup\bigcup_i\opsin{\typ_i}}
%\eqitem{\opsin{\exIopO}}
%       {\opsin{\typ}}
%\eqitem{\opsin{\exIO}}
%       {\opsin{\typ}\cup\opsin{\expr}}
%\eqitem{\opsin{\projO}}
%       {\opsin{\expr}\cup\bigcup_i\opsin{\typ_i}}
%\eqitem{\opsin{\recopO}}
%       {\bigcup_i\opsin{\typ_i}}
%\eqitem{\opsin{\rectO}}
%       {\bigcup_i\opsin{\typ_i}\cup\opsin{\expr_i}}
%\eqitem{\opsin{\tupleO}}
%       {\bigcup_i\opsin{\typ_i}\cup\opsin{\expr_i}}
%\end{eqlist}
%\]
%\end{theorem}

\begin{theorem}\label{thm-esbs-abbrev}
\[
\begin{eqlistC}
\eqitem{\esbsO{\true}}
       {\true}
\eqitem{\esbsO{\false}}
       {\false}
\eqitem{\esbsO{\negaop}}
       {\negaop}
\eqitem{\esbsO{(\conjO)}}
       {\conj{\esbsO{\expr_1}}{\esbsO{\expr_2}}}
\eqitem{\esbsO{(\disjO)}}
       {\disj{\esbsO{\expr_1}}{\esbsO{\expr_2}}}
\eqitem{\esbsO{(\implO)}}
       {\impl{\esbsO{\expr_1}}{\esbsO{\expr_2}}}
\eqitem{\esbsO{\iiffop}}
       {\iiffop}
\eqitem{\esbsO{(\iiffO)}}
       {\iiff{\esbsO{\expr_1}}{\esbsO{\expr_2}}}
\eqitem{\esbsO{(\neeqO)}}
       {\neeq{\esbsO{\expr_1}}{\esbsO{\expr_2}}}
\eqitem{\esbsO{(\descO)}}
       {\cond{\varI=\var}{\descO}{\desc{\var}{\typ}{\esbsO{\expr}}}}
\eqitem{\esbsO{\faopO}}
       {\faopO}
\eqitem{\esbsO{(\faO)}}
       {\cond{\varI=\var}{\faO}{\fa{\var}{\typ}{\esbsO{\expr}}}}
\eqitem{\esbsO{(\faS{\seqFT{\bnd{\var_1}{\typ_1}}{\bnd{\var_n}{\typ_n}}}{\expr})}}
       {\cond{\varI\in\varS}
             {\faS{\seqFT{\bnd{\var_1}{\typ_1}}{\bnd{\var_n}{\typ_n}}}{\expr}}
             {\faS{\seqFT{\bnd{\var_1}{\typ_1}}{\bnd{\var_n}{\typ_n}}}
                  {\esbsO{\expr}}}}
\eqitem{\esbsO{(\fa{\varS}{\typS}{\expr})}}
       {\cond{\varI\in\varS}
             {\fa{\varS}{\typS}{\expr}}
             {\fa{\varS}{\typS}{\esbsO{\expr}}}}
\eqitem{\esbsO{\exopO}}
       {\exopO}
\eqitem{\esbsO{(\exO)}}
       {\cond{\varI=\var}{\exO}{\ex{\var}{\typ}{\esbsO{\expr}}}}
\eqitem{\esbsO{(\exS{\seqFT{\bnd{\var_1}{\typ_1}}{\bnd{\var_n}{\typ_n}}}{\expr})}}
       {\cond{\varI\in\varS}
             {\exS{\seqFT{\bnd{\var_1}{\typ_1}}{\bnd{\var_n}{\typ_n}}}{\expr}}
             {\exS{\seqFT{\bnd{\var_1}{\typ_1}}{\bnd{\var_n}{\typ_n}}}
                  {\esbsO{\expr}}}}
\eqitem{\esbsO{(\ex{\varS}{\typS}{\expr})}}
       {\cond{\varI\in\varS}
             {\ex{\varS}{\typS}{\expr}}
             {\ex{\varS}{\typS}{\esbsO{\expr}}}}
\eqitem{\esbsO{\exIopO}}
       {\exIopO}
\eqitem{\esbsO{(\exIO)}}
       {\cond{\varI=\var}{\exIO}{\exI{\var}{\typ}{\esbsO{\expr}}}}
\eqitem{\esbsO{(\projO)}}
       {\proj{\esbsO{\expr}}{\fnam}}
\end{eqlistC}
\]
\end{theorem}

\begin{theorem}\label{thm-cvarv-abbrev}
\[
\begin{eqlist}
\eqitem{\cvarvO{\true}}
       {\emptyset}
\eqitem{\cvarvO{\false}}
       {\emptyset}
\eqitem{\cvarvO{\negaop}}
       {\emptyset}
\eqitem{\cvarvO{\conjO}}
       {\cvarvO{\expr_1}\cup\cvarvO{\expr_2}}
\eqitem{\cvarvO{\disjO}}
       {\cvarvO{\expr_1}\cup\cvarvO{\expr_2}}
\eqitem{\cvarvO{\implO}}
       {\cvarvO{\expr_1}\cup\cvarvO{\expr_2}}
\eqitem{\cvarvO{\iiffop}}
       {\emptyset}
\eqitem{\cvarvO{\iiffO}}
       {\cvarvO{\expr_1}\cup\cvarvO{\expr_2}}
\eqitem{\cvarvO{\neeqO}}
       {\cvarvO{\expr_1}\cup\cvarvO{\expr_2}}
\eqitem{\cvarvO{\descO}}
       {\cond{\varI\in\efvar{\expr}-\setI{\var}}
             {\setI{\var}\cup\cvarvO{\expr}}
             {\emptyset}}
\eqitem{\cvarvO{\faopO}}
       {\emptyset}
\eqitem{\cvarvO{\faO}}
       {\cond{\varI\in\efvar{\expr}-\setI{\var}}
             {\setI{\var}\cup\cvarvO{\expr}}
             {\emptyset}}
\eqitem{\cvarvO{\faS{\seqFT{\bnd{\var_1}{\typ_1}}{\bnd{\var_n}{\typ_n}}}{\expr}}}
       {\cond{\varI\in\efvar{\expr}-\varS}
             {\varS\cup\cvarvO{\expr}}
             {\emptyset}}
\eqitem{\cvarvO{\fa{\varS}{\typS}{\expr}}}
       {\cond{\varI\in\efvar{\expr}-\varS}
             {\varS\cup\cvarvO{\expr}}
             {\emptyset}}
\eqitem{\cvarvO{\exopO}}
       {\emptyset}
\eqitem{\cvarvO{\exO}}
       {\cond{\varI\in\efvar{\expr}-\setI{\var}}
             {\setI{\var}\cup\cvarvO{\expr}}
             {\emptyset}}
\eqitem{\cvarvO{\exS{\seqFT{\bnd{\var_1}{\typ_1}}{\bnd{\var_n}{\typ_n}}}{\expr}}}
       {\cond{\varI\in\efvar{\expr}-\varS}
             {\varS\cup\cvarvO{\expr}}
             {\emptyset}}
\eqitem{\cvarvO{\ex{\varS}{\typS}{\expr}}}
       {\cond{\varI\in\efvar{\expr}-\varS}
             {\varS\cup\cvarvO{\expr}}
             {\emptyset}}
\eqitem{\cvarvO{\exIopO}}
       {\emptyset}
\eqitem{\cvarvO{\exIO}}
       {\cond{\varI\in\efvar{\expr}-\setI{\var}}
             {\setI{\var}\cup\cvarvO{\expr}}
             {\emptyset}}
\eqitem{\cvarvO{\projO}}
       {\cvarvO{\expr}}
\end{eqlist}
\]
\end{theorem}

\begin{theorem}\label{thm-tsbs-abbrev}
\[
\begin{eqlistC}
\eqitem{\tsbsO{\true}}
       {\true}
\eqitem{\tsbsO{\false}}
       {\false}
\eqitem{\tsbsO{\negaop}}
       {\negaop}
\eqitem{\tsbsO{(\conjO)}}
       {\conj{\tsbsO{\expr_1}}{\tsbsO{\expr_2}}}
\eqitem{\tsbsO{(\disjO)}}
       {\disj{\tsbsO{\expr_1}}{\tsbsO{\expr_2}}}
\eqitem{\tsbsO{(\implO)}}
       {\impl{\tsbsO{\expr_1}}{\tsbsO{\expr_2}}}
\eqitem{\tsbsO{\iiffop}}
       {\emptyset}
\eqitem{\tsbsO{(\iiffO)}}
       {\iiff{\tsbsO{\expr_1}}{\tsbsO{\expr_2}}}
\eqitem{\tsbsO{(\neeqO)}}
       {\neeq{\tsbsO{\expr_1}}{\tsbsO{\expr_2}}}
\eqitem{\tsbsO{(\descO)}}
       {\desc{\var}{\tsbsO{\typ}}{\tsbsO{\expr}}}
\eqitem{\tsbsO{\faopO}}
       {\faop{\tsbsO{\typ}}}
\eqitem{\tsbsO{(\faO)}}
       {\fa{\var}{\tsbsO{\typ}}{\tsbsO{\expr}}}
\eqitem{\tsbsO{(\faS{\seqFT{\bnd{\var_1}{\typ_1}}{\bnd{\var_n}{\typ_n}}}{\expr})}}
       {\faS{\seqFT{\bnd{\var_1}{\tsbsO{\typ_1}}}{\bnd{\var_n}{\tsbsO{\typ_n}}}}
            {\tsbsO{\expr}}}
\eqitem{\tsbsO{(\fa{\varS}{\typS}{\expr})}}
       {\fa{\varS}{\tsbsO{\typS}}{\tsbsO{\expr}}}
\eqitem{\tsbsO{\exopO}}
       {\exop{\tsbsO{\typ}}}
\eqitem{\tsbsO{(\exO)}}
       {\ex{\var}{\tsbsO{\typ}}{\tsbsO{\expr}}}
\eqitem{\tsbsO{(\exS{\seqFT{\bnd{\var_1}{\typ_1}}{\bnd{\var_n}{\typ_n}}}{\expr})}}
       {\exS{\seqFT{\bnd{\var_1}{\tsbsO{\typ_1}}}{\bnd{\var_n}{\tsbsO{\typ_n}}}}
            {\tsbsO{\expr}}}
\eqitem{\tsbsO{(\ex{\varS}{\typS}{\expr})}}
       {\ex{\varS}{\tsbsO{\typS}}{\tsbsO{\expr}}}
\eqitem{\tsbsO{\exIopO}}
       {\exIop{\tsbsO{\typ}}}
\eqitem{\tsbsO{(\exIO)}}
       {\exI{\var}{\tsbsO{\typ}}{\tsbsO{\expr}}}
\eqitem{\tsbsO{(\projO)}}
       {\proj{\tsbsO{\expr}}{\fnam}}
\end{eqlistC}
\]
\end{theorem}

\begin{theorem}\label{thm-ftvar-abbrev}
\[
\begin{eqlist}
\eqitem{\ftvar{\true}}
       {\emptyset}
\eqitem{\ftvar{\false}}
       {\emptyset}
\eqitem{\ftvar{\negaop}}
       {\emptyset}
\eqitem{\ftvar{\conjO}}
       {\ftvar{\expr_1}\cup\ftvar{\expr_2}}
\eqitem{\ftvar{\disjO}}
       {\ftvar{\expr_1}\cup\ftvar{\expr_2}}
\eqitem{\ftvar{\implO}}
       {\ftvar{\expr_1}\cup\ftvar{\expr_2}}
\eqitem{\ftvar{\iiffop}}
       {\emptyset}
\eqitem{\ftvar{\iiffO}}
       {\ftvar{\expr_1}\cup\ftvar{\expr_2}}
\eqitem{\ftvar{\neeqO}}
       {\ftvar{\expr_1}\cup\ftvar{\expr_2}}
\eqitem{\ftvar{\descO}}
       {\ftvar{\typ}\cup\ftvar{\expr}}
\eqitem{\ftvar{\faopO}}
       {\ftvar{\typ}}
\eqitem{\ftvar{\faO}}
       {\ftvar{\typ}\cup\ftvar{\expr}}
\eqitem{\ftvar{\faS{\seqFT{\bnd{\var_1}{\typ_1}}{\bnd{\var_n}{\typ_n}}}{\expr}}}
       {\ftvar{\expr}\cup\bigcup_i\ftvar{\typ_i}}
\eqitem{\ftvar{\fa{\varS}{\typS}{\expr}}}
       {\ftvar{\expr}\cup\bigcup_i\ftvar{\typ_i}}
\eqitem{\ftvar{\exopO}}
       {\ftvar{\typ}}
\eqitem{\ftvar{\exO}}
       {\ftvar{\typ}\cup\ftvar{\expr}}
\eqitem{\ftvar{\exS{\seqFT{\bnd{\var_1}{\typ_1}}{\bnd{\var_n}{\typ_n}}}{\expr}}}
       {\ftvar{\expr}\cup\bigcup_i\ftvar{\typ_i}}
\eqitem{\ftvar{\ex{\varS}{\typS}{\expr}}}
       {\ftvar{\expr}\cup\bigcup_i\ftvar{\typ_i}}
\eqitem{\ftvar{\exIopO}}
       {\ftvar{\typ}}
\eqitem{\ftvar{\exIO}}
       {\ftvar{\typ}\cup\ftvar{\expr}}
\eqitem{\ftvar{\projO}}
       {\ftvar{\expr}\cup\bigcup_i\ftvar{\typ_i}}
\end{eqlist}
\]
\end{theorem}

\subsection{Proof-theoretical properties}

Any prefix of a well-formed context is itself well-formed:

\begin{theorem}\label{thm-cxprefix}
\[
\cxwf{\seqII{\cx_1}{\cx_2}}\ \ \IMPLIES\ \ \cxwf{\cx_1}
\]
\end{theorem}

If we derive a judgement with some context, that context is well-formed:

\begin{theorem}\label{thm-cxwf}
\[
\jcxdots{\cx}\ \ \IMPLIES\ \ \cxwf{\cx}
\]
\end{theorem}

The following four theorems say that well-formed contexts have no duplicate
types, ops, and (type) variables:

\begin{theorem}\label{thm-cx-uniq-typ}
\[
\cxwf{\cx_1,\tdecO,\cx_2}\IMPLIES\tnam\not\in\cxtnam{\cx_1,\cx_2}
\]
\end{theorem}

\begin{theorem}\label{thm-cx-uniq-op}
\[
\cxwf{\cx_1,\odecO,\cx_2}\IMPLIES\onam\not\in\cxonam{\cx_1,\cx_2}
\]
\end{theorem}

\begin{theorem}\label{thm-cx-uniq-tvar}
\[
\cxwf{\cx_1,\tvdecO,\cx_2}\IMPLIES\tvar\not\in\cxtvar{\cx_1,\cx_2}
\]
\end{theorem}

\begin{theorem}\label{thm-cx-uniq-var}
\[
\cxwf{\cx_1,\vdecO,\cx_2}\IMPLIES\var\not\in\cxvar{\cx_1,\cx_2}
\]
\end{theorem}

The type of an op declared in a well-formed context is well-formed in the
prefix of the context that precedes the op, extended with the type variables
over which the op is polymorphic:

\begin{theorem}\label{thm-op-type-wf}
\[
\cxwf{\cx_1,\odecO,\cx_2}\ \ \IMPLIES\ \ 
\isty{\snoc{\cx_1}{\tvdec{\tvarS}}}{\typ}
\]
\end{theorem}

A type definition in a well-formed context defines a type previously declared
in the context with the right arity, and the defining type is well-formed in
the prefix of the context that precedes the type definition, extended with the
type variable arguments of the defined type:

\begin{theorem}\label{thm-tdef-wf}
\[
\cxwf{\cx_1,\tdefO,\cx_2}
\ \ \IMPLIES\ \
\tdec{\tnam}{\seqlen{\tvarS}}\in\cx_1
\ \ \AND\ \
\isty{\snoc{\cx_1}{\tvdec{\tvarS}}}{\typ}
\]
%\begin{array}{rl}
%\cxwf{\cx_1,\tdefO,\cx_2}
%\IMPLIES & \tdec{\tnam}{\seqlen{\tvarS}}\in\cx_1 \\
%\AND     & \isty{\snoc{\cx_1}{\tvdec{\tvarS}}}{\typ}
%\end{array}
\end{theorem}

An axiom declared in a well-formed context is well-typed (with type $\bool$)
in the prefix of the context that precedes the axiom, extended with the type
variables over which the axiom is polymorphic:

\begin{theorem}\label{thm-ax-wt}
\[
\cxwf{\cx_1,\axO,\cx_2}\ \ \IMPLIES\ \
\hasty{\snoc{\cx_1}{\tvdec{\tvarS}}}{\expr}{\bool}
\]
\end{theorem}

A lemma declared in a well-formed context is a theorem in the prefix of the
context that precedes the lemma, extended with the type variables over which
the lemma is polymorphic:

\begin{theorem}\label{thm-lem-theo}
\[
\cxwf{\cx_1,\lemO,\cx_2}\ \ \IMPLIES\ \
\theo{\snoc{\cx_1}{\tvdec{\tvarS}}}{\expr}
\]
\end{theorem}

The type of a variable declared in a well-formed context is well-formed in the
prefix of the context that precedes the variable declaration:

\begin{theorem}\label{thm-var-type-wf}
\[
\cxwf{\cx_1,\vdecO,\cx_2}\ \ \IMPLIES\ \ \isty{\cx_1}{\typ}
\]
\end{theorem}

A well-formed type variable is declared in the context:

\begin{theorem}\label{thm-tvar-inv}
\[
\istyO{\tvar}\ \ \IMPLIES\ \ \tvar\in\cxtvarO
\]
\end{theorem}

The type name of a well-formed type instance is declared in the context, with
an arity that matches the type arguments, and each type argument is
well-formed:

\begin{theorem}\label{thm-tinst-inv}
\[
\istyO{\tinstO}
\ \ \IMPLIES\ \
\tdec{\tnam}{\seqlen{\typS}}\in\cx
\ \ \AND\ \
(\FORALL{i}{\istyO{\typ_i}})
\]
\end{theorem}

The domain and range types of a well-formed arrow type are well-formed:

\begin{theorem}\label{thm-tarr-inv}
\[
\istyO{\tarrO}\ \ \IMPLIES\ \ \istyO{\typ_1}\ \ \AND\ \ \istyO{\typ_2}
\]
\end{theorem}

The component types of a well-formed record type are well-formed:
%The component types of a well-formed record or sum type are well-formed:

\begin{theorem}\label{thm-trec-inv}
\[
\istyO{\trecO}\ \ \IMPLIES\ \ (\FORALL{i}{\istyO{\typ_i}})
\]
\end{theorem}

%\begin{theorem}\label{thm-tsum-inv}
%\[
%\istyO{\tsumO}\ \ \IMPLIES\ \ (\FORALL{i}{\istyO{\typ_i}})
%\]
%\end{theorem}

The predicate of a restriction type is well-typed, with the expected type:

\begin{theorem}\label{thm-tspred-wt}
\[
\istyO{\tsubO}\ \ \IMPLIES\ \ \hastyO{\tspred}{\tarr{\typ}{\bool}}
\]
\end{theorem}

The predicate of a well-formed restriction type has no free variables:

\begin{theorem}\label{thm-no-free-vars-in-types}
\[
\istyO{\tsubO}\ \ \IMPLIES\ \ \efvar{\tspred}=\emptyset
\]
\end{theorem}

For each well-typed application, the first expression (function) has an arrow
type and the second expression (argument) has the domain type:

\begin{theorem}\label{thm-eapp-inv}
\[
\hastyOany{\appO}
\ \ \IMPLIES\ \ 
\EXISTS{\typ_1,\typ_2}
       {\hastyO{\expr_1}{\tarrO}\ \AND\ \hastyO{\expr_2}{\typ_1}}
\]
\end{theorem}

A well-typed abstraction has an arrow type whose domain is the type of its
bound variable:

\begin{theorem}\label{thm-abs-arrty}
\[
\hastyOany{\absO}
\ \ \IMPLIES\ \
\EXISTS{\typ'}{\hastyO{\absO}{\tarr{\typ}{\typ'}}}
\]
\end{theorem}

The subtype predicates that appear in (provable) subtype judgements have no
free variables:

\begin{theorem}\label{thm-subpred-no-free-vars}
\[
\issubO{\typ_1}{\tspred}{\typ_2}\IMPLIES\efvar{\tspred}=\emptyset
\]
\end{theorem}

All the free variables in a well-typed expression or a theorem are declared in
the context, and all the free variables in an axiom or lemma of a well-formed
context are declared before the axiom or lemma:

\begin{theorem}\label{thm-free-var-in-cx}
\[
\begin{array}{l@{\ \ \IMPLIES\ \ }l}
\hastyO{\expr}{\typ}     & \efvar{\expr}\subseteq\cxvarO       \\
\theoO{\expr}            & \efvar{\expr}\subseteq\cxvarO       \\
\cxwf{\cx_1,\axO, \cx_2} & \efvar{\expr}\subseteq\cxvar{\cx_1} \\
\cxwf{\cx_1,\lemO,\cx_2} & \efvar{\expr}\subseteq\cxvar{\cx_1}
\end{array}
\]
\end{theorem}

%All the free variables in an axiom or lemma in a well-formed context are
%declared before the axiom or lemma:

%\begin{theorem}\label{thm-free-var-in-prev-cx}
%\[
%\begin{array}{l@{\ \ \IMPLIES\ \ }l}
%\cxwf{\cx_1,\axO, \cx_2} & \efvar{\expr}\subseteq\cxvar{\cx_1} \\
%\cxwf{\cx_1,\lemO,\cx_2} & \efvar{\expr}\subseteq\cxvar{\cx_1}
%\end{array}
%\]
%\end{theorem}

All the free type variables in the types and expressions to the right of a
provable judgement $\jcxdots{\cx}$ are declared in $\cx$, and all the free
type variables in a context element of a well-formed context are declared
before the context element:

\begin{theorem}\label{thm-free-tvar-in-cx}
\[
\begin{array}{l@{\ \ \IMPLIES\ \ }l}
\istyO{\typ} &
 \cxtvarO\supseteq\ftvar{\typ} \\
\teqO{\typ_1}{\typ_2} &
 \cxtvarO\supseteq\ftvar{\typ_1}\cup\ftvar{\typ_2} \\
\issubO{\typ_1}{\tspred}{\typ_2} &
 \cxtvarO\supseteq\ftvar{\typ_1}\cup\ftvar{\tspred}\cup\ftvar{\typ_2} \\
\hastyO{\expr}{\typ} &
 \cxtvarO\supseteq\ftvar{\expr}\cup\ftvar{\typ} \\
\theoO{\expr} &
 \cxtvarO\supseteq\ftvar{\expr} \\
\cxwf{\cx_1,\cxel,\cx_2} &
 \cxtvar{\cx_1}\supseteq\ftvar{\cxel}
\end{array}
\]
\end{theorem}

The derivation of a judgement does not depend on the choice of the names for
the variables in the derivation. So, if we replace a variable declared in the
context of the judgement with a fresh one\footnote{The adjective ``fresh''
means that the variable does not occur anywhere in the (judgements that
comprise the) derivation that is under discussion. This notion could be made
more formal, but it should be sufficiently clear as stated.} and perform the
substitution in the rest of the judgement accordingly, the judgement is still
provable:

\begin{theorem}\label{thm-deriv-rename-var}
\[
\begin{array}{@{\fresh{\varI'}\ \ \AND\ \ }l@{\ \ \IMPLIES\ \ }l}
\cxwf{\cxv} &
 \cxwf{\cxvI} \\
\isty{\cxv}{\typ} &
 \isty{\cxvI}{\typ} \\
\teq{\cxv}{\typ_1}{\typ_2} &
 \teq{\cxvI}{\typ_1}{\typ_2} \\
\issub{\cxv}{\typ_1}{\tspred}{\typ_2} &
 \issub{\cxvI}{\typ_1}{\tspred}{\typ_2} \\
\hasty{\cxv}{\expr}{\typ} &
 \hasty{\cxvI}{\esbsren{\expr}}{\typ} \\
\theo{\cxv}{\expr} &
 \theo{\cxvI}{\esbsren{\expr}}
\end{array}
\]
\end{theorem}

The \MS\ logic is monotonic, in the sense that anything can be derived in a
bigger context that can be derived in a smaller context:

\begin{theorem}\label{thm-mono}
\[
\cxwf{\cx'}\ \ \AND\ \ \cx\subseteq\cx'\ \ \IMPLIES\ \
\left(
\begin{array}{l@{\ \ \IMPLIES\ \ }l}
\istyO{\typ}                     & \isty{\cx'}{\typ}                     \\
\teqO{\typ_1}{\typ_2}            & \teq{\cx'}{\typ_1}{\typ_2}            \\
\issubO{\typ_1}{\tspred}{\typ_2} & \issub{\cx'}{\typ_1}{\tspred}{\typ_2} \\
\hastyO{\expr}{\typ}             & \hasty{\cx'}{\expr}{\typ}             \\
\theoO{\expr}                    & \theo{\cx'}{\expr}
\end{array}
\right)
\]
\end{theorem}

Note that the monotonicity theorem above also applies to the case in which
$\cx'$ is a (well-formed) permutation of $\cx$ (it is still the case that
$\cx\subseteq\cx'$). In other words, anything that can be derived in a context
can be also derived in any well-formed permutation of that context.

The following theorem says that if we derive a judgement
$\jcxdots{\cx_1,\tvdec{\tvarIS},\cx_3}$, then we can substitute the $\tvarIS$
with well-formed types $\typIS$, under certain conditions:

\begin{theorem}\label{thm-tsbs-jdg}
\[
\begin{array}{l}
\FORALL{i}{\isty{\cx_1,\cx_2}{\typI_i}}
\\
\tsbslashok{\cx_3}{\tvarIS}{\typIS}
\\
\cxwf{\cx_1,\cx_2,\tsbslash{\cx_3}{\tvarIS}{\typIS}}
\\
\ \ \IMPLIES\\
\left(
\begin{array}{l@{\ \ \IMPLIES\ \ }l}
\isty{\cx_1,\tvdec{\tvarIS},\cx_3}{\typ} &
\isty{\cx_1,\cx_2,\tsbslash{\cx_3}{\tvarIS}{\typIS}}
     {\tsbslash{\typ}{\tvarIS}{\typIS}} \\
\teq{\cx_1,\tvdec{\tvarIS},\cx_3}{\typ_1}{\typ_2} &
\teq{\cx_1,\cx_2,\tsbslash{\cx_3}{\tvarIS}{\typIS}}
    {\tsbslash{\typ_1}{\tvarIS}{\typIS}}
    {\tsbslash{\typ_2}{\tvarIS}{\typIS}} \\
\issub{\cx_1,\tvdec{\tvarIS},\cx_3}{\typ_1}{\tspred}{\typ_2} &
\issub{\cx_1,\cx_2,\tsbslash{\cx_3}{\tvarIS}{\typIS}}
      {\tsbslash{\typ_1}{\tvarIS}{\typIS}}
      {\tsbslash{\tspred}{\tvarIS}{\typIS}}
      {\tsbslash{\typ_2}{\tvarIS}{\typIS}} \\
\hasty{\cx_1,\tvdec{\tvarIS},\cx_3}{\expr}{\typ} &
\hasty{\cx_1,\cx_2,\tsbslash{\cx_3}{\tvarIS}{\typIS}}
      {\tsbslash{\expr}{\tvarIS}{\typIS}}
      {\tsbslash{\typ}{\tvarIS}{\typIS}} \\
\theo{\cx_1,\tvdec{\tvarIS},\cx_3}{\expr} &
\theo{\cx_1,\cx_2,\tsbslash{\cx_3}{\tvarIS}{\typIS}}
     {\tsbslash{\expr}{\tvarIS}{\typIS}}
\end{array}
\right)
\end{array}
\]
\end{theorem}

The following theorems prove derived well-formedness rules for contexts:

\begin{theorem}\label{thm-cxvdecI}
{\rm
\[
\rruleII
 {\istyO{\typ}}
 {\var\not\in\cxvarO}
 {\cxwf{\snoc{\cx}{\vdecO}}}
 {\RcxvdecI}
\]
}
\end{theorem}

\begin{theorem}\label{thm-cxvdecII}
{\rm
\[
\rruleIII
 {\istyO{\typ}}
 {\cxwf{\cx,\cx'}}
 {\var\not\in\cxvar{\cx,\cx'}}
 {\cxwf{\cx,\cx',\vdecO}}
 {\RcxvdecII}
\]
}
\end{theorem}

\begin{theorem}\label{thm-cxvdecbool}
{\rm
\[
\rruleII
 {\cxwfO}
 {\var\not\in\cxvarO}
 {\cxwf{\snoc{\cx}{\vdec{\var}{\bool}}}}
 {\Rcxvdecbool}
\]
}
\end{theorem}

\begin{theorem}\label{thm-axmono}
{\rm
\[
\rruleI
 {\hastyO{\expr}{\bool}}
 {\cxwf{\snoc{\cx}{\axM{\expr}}}}
 {\Rcxaxmono}
\]
}
\end{theorem}

The following theorems prove derived well-typedness rules for expressions,
including the abbreviations defined in \secref{syntax}:

\begin{theorem}\label{thm-eabsbool}
{\rm
\[
\rruleI
 {\hasty{\cx,\vdecO}{\expr}{\bool}}
 {\hastyO{\absO}{\tarr{\typ}{\bool}}}
 {\Reabsbool}
\]
}
\end{theorem}

\begin{theorem}\label{thm-eid}
{\rm
\[
\rruleI
 {\istyO{\typ}}
 {\hastyO{\abs{\var}{\typ}{\var}}{\tarr{\typ}{\typ}}}
 {\Reid}
\]
}
\end{theorem}

\begin{theorem}\label{thm-eidbool}
{\rm
\[
\rruleI
 {\cxwfO}
 {\hastyO{\abs{\var}{\bool}{\var}}{\tarr{\bool}{\bool}}}
 {\Reidbool}
\]
}
\end{theorem}

\begin{theorem}\label{thm-etrue}
{\rm
\[
\rruleI
 {\cxwfO}
 {\hastyO{\true}{\bool}}
 {\Retrue}
\]
}
\end{theorem}

\begin{theorem}\label{thm-econsttrue}
{\rm
\[
\rruleI
 {\istyO{\typ}}
 {\hastyO{\abs{\var}{\typ}{\true}}{\tarr{\typ}{\bool}}}
 {\Reconsttrue}
\]
}
\end{theorem}

\begin{theorem}\label{thm-efalse}
{\rm
\[
\rruleI
 {\cxwfO}
 {\hastyO{\false}{\bool}}
 {\Refalse}
\]
}
\end{theorem}

\begin{theorem}\label{thm-enot}
{\rm
\[
\rruleI
 {\cxwfO}
 {\hastyO{\negaop}{\tarr{\bool}{\bool}}}
 {\Renot}
\]
}
\end{theorem}

In the proof of the previous theorem, we use \ReifO, not \Reif. An attempt to
use \Reif\ causes a circularity, where in order to prove that $\negaop$ is
well-typed one has to first prove that $\negaop$ is well-typed. The reader can
try a backward derivation of
$\hasty{\snoc{\cx}{\vdec{\varfx}{\bool}}}{\iif{\varfx}{\false}{\true}}{\bool}$
where the first rule applied backwards is \Reif: while the first two premises
(i.e.\ $\hasty{\snoc{\cx}{\vdec{\varfx}{\bool}}}{\varfx}{\bool}$ and
$\hasty{\snoc{\snoc{\cx}{\vdec{\varfx}{\bool}}}{\axM{\varfx}}}{\false}{\bool}$)
can be derived with ease, the third premise (i.e.\
$\hasty{\snoc{\snoc{\cx}{\vdec{\varfx}{\bool}}}{\axM{\nega{\varfx}}}}
{\true}{\bool}$) requires proving
$\cxwf{\snoc{\snoc{\cx}{\vdec{\varfx}{\bool}}}{\axM{\nega{\varfx}}}}$ (so that
we can use \Retrue), which requires proving
$\hasty{\snoc{\cx}{\vdec{\varfx}{\bool}}}{\nega{\varfx}}{\bool}$ (so that we
can use \Rcxax), which requires proving
$\hasty{\snoc{\cx}{\vdec{\varfx}{\bool}}}{\negaop}{\tarr{\bool}{\bool}}$ (so
that we can use \Reapp), which causes a circularity. The arguments just given
do not constitute a formal proof of the necessity of rule \ReifO, but they do
suggest te existence of such a proof.

\begin{theorem}\label{thm-enega}
{\rm
\[
\rruleI
 {\hastyO{\expr}{\bool}}
 {\hastyO{\negaO}{\bool}}
 {\Renega}
\]
}
\end{theorem}

\begin{theorem}\label{thm-econj}
{\rm
\[
\rruleII
 {\hastyO{\expr_1}{\bool}}
 {\hasty{\snoc{\cx}{\axM{\expr_1}}}{\expr_2}{\bool}}
 {\hastyO{\conjO}{\bool}}
 {\Reconj}
\]
}
\end{theorem}

\begin{theorem}\label{thm-econjO}
{\rm
\[
\rruleII
 {\hastyO{\expr_1}{\bool}}
 {\hastyO{\expr_2}{\bool}}
 {\hastyO{\conjO}{\bool}}
 {\ReconjO}
\]
}
\end{theorem}

\begin{theorem}\label{thm-edisj}
{\rm
\[
\rruleII
 {\hastyO{\expr_1}{\bool}}
 {\hasty{\snoc{\cx}{\axM{\nega{\expr_1}}}}{\expr_2}{\bool}}
 {\hastyO{\disjO}{\bool}}
 {\Redisj}
\]
}
\end{theorem}

\begin{theorem}\label{thm-edisjO}
{\rm
\[
\rruleII
 {\hastyO{\expr_1}{\bool}}
 {\hastyO{\expr_2}{\bool}}
 {\hastyO{\disjO}{\bool}}
 {\RedisjO}
\]
}
\end{theorem}

\begin{theorem}\label{thm-eimpl}
{\rm
\[
\rruleII
 {\hastyO{\expr_1}{\bool}}
 {\hasty{\snoc{\cx}{\axM{\expr_1}}}{\expr_2}{\bool}}
 {\hastyO{\implO}{\bool}}
 {\Reimpl}
\]
}
\end{theorem}

\begin{theorem}\label{thm-eimplO}
{\rm
\[
\rruleII
 {\hastyO{\expr_1}{\bool}}
 {\hastyO{\expr_2}{\bool}}
 {\hastyO{\implO}{\bool}}
 {\ReimplO}
\]
}
\end{theorem}

\begin{theorem}\label{thm-eiff}
{\rm
\[
\rruleI
 {\cxwfO}
 {\hastyO{\iiffop}{\tarr{\bool}{\tarr{\bool}{\bool}}}}
 {\Reiff}
\]
}
\end{theorem}

\begin{theorem}\label{thm-ecoimpl}
{\rm
\[
\rruleII
 {\hastyO{\expr_1}{\bool}}
 {\hastyO{\expr_2}{\bool}}
 {\hastyO{\iiffO}{\bool}}
 {\Recoimpl}
\]
}
\end{theorem}

\begin{theorem}\label{thm-eneq}
{\rm
\[
\rruleII
 {\hastyO{\expr_1}{\typ}}
 {\hastyO{\expr_2}{\typ}}
 {\hastyO{\neeqO}{\bool}}
 {\Reneq}
\]
}
\end{theorem}

\begin{theorem}\label{thm-efaop}
{\rm
\[
\rruleI
 {\istyO{\typ}}
 {\hastyO{\faopO}{\tarr{(\tarr{\typ}{\bool})}{\bool}}}
 {\Refaop}
\]
}
\end{theorem}

\begin{theorem}\label{thm-efa}
{\rm
\[
\rruleI
 {\hasty{\snoc{\cx}{\vdecO}}{\expr}{\bool}}
 {\hastyO{\faO}{\bool}}
 {\Refa}
\]
}
\end{theorem}

\begin{theorem}\label{thm-eexop}
{\rm
\[
\rruleI
 {\istyO{\typ}}
 {\hastyO{\exopO}{\tarr{(\tarr{\typ}{\bool})}{\bool}}}
 {\Reexop}
\]
}
\end{theorem}

\begin{theorem}\label{thm-eex}
{\rm
\[
\rruleI
 {\hasty{\snoc{\cx}{\vdecO}}{\expr}{\bool}}
 {\hastyO{\exO}{\bool}}
 {\Reex}
\]
}
\end{theorem}

\begin{theorem}\label{thm-eexIop}
{\rm
\[
\rruleI
 {\istyO{\typ}}
 {\hastyO{\exIopO}{\tarr{(\tarr{\typ}{\bool})}{\bool}}}
 {\ReexIop}
\]
}
\end{theorem}

\begin{theorem}\label{thm-eexI}
{\rm
\[
\rruleI
 {\hasty{\snoc{\cx}{\vdecO}}{\expr}{\bool}}
 {\hastyO{\exIO}{\bool}}
 {\ReexI}
\]
}
\end{theorem}

\begin{theorem}\label{thm-efaalphaO}
{\rm
\[
\rruleII
 {\hasty{\cx,\vdecO}{\expr}{\bool}}
 {\var'\not\in\efvar{\expr}\cup\cvarv{\expr}{\var}}
 {\hastyO{\fa{\var'}{\typ}{\esbs{\expr}{\var}{\var'}}}{\bool}}
 {\RefaalphaO}
\]
}
\end{theorem}

\begin{theorem}\label{thm-edotO}
{\rm
\[
\rruleII
 {\istyO{\trecO}}
 {\hastyO{\expr}{\trecO}}
 {\hastyO{\proj{\expr}{\fnam_j}}{\typ_j}}
 {\RedotO}
\]
}
\end{theorem}

Equivalent types are well-formed:

\begin{theorem}\label{thm-equiv-types-wf}
\[
\teqO{\typ_1}{\typ_2}\ \ \IMPLIES\ \ \istyO{\typ_1}\ \ \AND\ \ \istyO{\typ_2}
\]
\end{theorem}

If an expression has a type $\typ$, it also has all types $\typ'$ that are
equivalent to $\typ$:

\begin{theorem}\label{thm-eteq}
{\rm
\[
\rruleII
 {\hastyO{\expr}{\typ}}
 {\teqO{\typ}{\typ'}}
 {\hastyO{\expr}{\typ'}}
 {\Reteq}
\]
}
\end{theorem}

If a type is a subtype of another one, both types are well-formed; in
addition, the predicate that defines the subtype is well-typed and is indeed a
predicate over the supertype:

\begin{theorem}\label{thm-subtyping-wf}
\[
\issubO{\typ_1}{\tspred}{\typ_2}
\ \ \IMPLIES\ \
\istyO{\typ_1}
\ \ \AND\ \
\istyO{\typ_2}
\ \ \AND\ \
\hastyO{\tspred}{\tarr{\typ_2}{\bool}}
\]
\end{theorem}

The type of a well-typed expression is well-formed:

\begin{theorem}\label{thm-exty-wf}
\[
\hastyO{\expr}{\typ}\ \ \IMPLIES\ \ \istyO{\typ}
\]
\end{theorem}

Using the previous theorem, we can eliminate the first premise from rule
\RedotO\ (however, note that \RedotO\ is used in the proof of
\thmref{thm-subtyping-wf}, which is proved by induction together with
\thmref{thm-exty-wf}):

\begin{theorem}\label{thm-edot}
{\rm
\[
\rruleI
 {\hastyO{\expr}{\trecO}}
 {\hastyO{\proj{\expr}{\fnam_j}}{\typ_j}}
 {\Redot}
\]
}
\end{theorem}

The expression $\true$ is a theorem:

\begin{theorem}\label{thm-thtrue}
{\rm
\[
\rruleI
 {\cxwfO}
 {\theoO{\true}}
 {\Rthtrue}
\]
}
\end{theorem}

Proving that an expression equals $\true$ implies that the expression is a
theorem:

\begin{theorem}\label{thm-theqtrue}
{\rm
\[
\rruleI
 {\theoO{\eq{\expr}{\true}}}
 {\theoO{\expr}}
 {\Rtheqtrue}
\]
}
\end{theorem}

A universally quantified theorem can be instantiated with any well-typed
expression whose type matches the universally quantified variable's and whose
substitution does not cause capture:

\begin{theorem}\label{thm-thfaelimO}
{\rm
\[
\rruleIV
 {\hastyO{\faO}{\bool}}
 {\theoO{\faO}}
 {\hastyO{\expr'}{\typ}}
 {\esbsok{\expr}{\var}{\expr'}}
 {\theoO{\esbs{\expr}{\var}{\expr'}}}
 {\RthfaelimO}
\]
}
\end{theorem}

%\begin{theorem}
%{\rm
%\[
%\rrule
%\]
%}
%\end{theorem}
